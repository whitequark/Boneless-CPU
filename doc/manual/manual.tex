\documentclass[12pt,a4paper]{article}
\usepackage[hmargin=3cm,vmargin=2.5cm]{geometry}
\usepackage[utf8]{inputenc}
\usepackage{import}
\usepackage{parskip}
\usepackage{color}
\usepackage[color,leftbars]{changebar}
\usepackage{ragged2e}
\usepackage{changepage}
\usepackage{alltt}
\renewcommand{\ttdefault}{txtt}
\usepackage{array}
\newcolumntype{L}[1]{>{\raggedright\let\newline\\\arraybackslash\hspace{0pt}}m{#1}}
\newcolumntype{C}[1]{>{\centering\let\newline\\\arraybackslash\hspace{0pt}}m{#1}}
\newcolumntype{R}[1]{>{\raggedleft\let\newline\\\arraybackslash\hspace{0pt}}m{#1}}
\usepackage{hyperref}
\hypersetup{
  colorlinks=true,
  linkcolor=blue,
  pdfstartview=Fit,
  pdfpagemode=UseOutlines,
}

\newcommand{\undefined}{\textbf{UNDEFINED}}
\newcommand{\unpredictable}{\textbf{UNPREDICTABLE}}
\uchyph=0

\newenvironment{notice}{
  \cbcolor{red}
  \par\textbf{Notice:}\cbstart\par
  \begin{adjustwidth}{10pt}{0pt}
}{
  \end{adjustwidth}
  \cbend
}

\begin{document}

\begin{titlepage}
  \centering
  \vspace*{6cm}
  \par{\Huge Boneless-III}
  \vspace{1cm}
  \par{\Huge Architecture Reference Manual}
\end{titlepage}

\begin{notice}
This document is a work in progress and subject to change without warning. However, the parts that are \textit{especially} subject to change carry a notice similar to this one.
\end{notice}

\pagebreak

\tableofcontents
\addcontentsline{toc}{section}{Table of Contents}
\pagebreak

\setcounter{tocdepth}{2}

\section{Introduction}
TBD
\pagebreak

\section{Guide to Instruction Set}

\subsection{Operation Syntax}
\def←{$\leftarrow$}

This document uses the following syntax and operators to describe the operation of each instruction.

\subsubsection{Undefined and Unpredictable Behavior}
To describe the boundaries of legal program behavior, this document uses the words \undefined{} and \unpredictable{}.

When execution encounters \unpredictable{} behavior, the implementation may perform any behavior, including but not limited to hanging and failing to continue execution. The resulting behavior may be different between executions even under the same circumstances.

Certain operations, including any operation with an \undefined{} input, will produce an undefined result. Reading a register whose value is currently \undefined{} may produce any bit pattern. Multiple consecutive reads of such a register may also produce different bit patterns on each read.

\subsubsection{Reference Operators}
The following operators reference parts of variables or the attached memory.

\begin{itemize}
  \item \texttt{opB} ← \texttt{opA}: Store \texttt{opA} into \texttt{opB}. If necessary, \texttt{opA} is implicitly zero-extended or truncated to match the length of \texttt{opB}.
  \item \texttt{op[b:a]}: Reference bits \texttt{a} through \texttt{b}, inclusive, of \texttt{op}.
  \item \texttt{mem[addr]}: Reference memory at address \texttt{addr}. The address is implicitly ANDed with \texttt{0xFFFF}.
  \item \texttt{ext[addr]}: Reference the external bus at address \texttt{addr}. The address is implicitly ANDed with \texttt{0xFFFF}.
  \item \texttt{\string{opA, opB\string}}: Concatenate the bits of \texttt{opA} and \texttt{opB}. \texttt{opA} makes the high-order bits of the result and \texttt{opB} makes the low-order bits.
  \item \texttt{opB\string{opA\string}}: Construct the result by repeating \texttt{opA} \texttt{opB} times.
\end{itemize}

\subsubsection{Arithmetic Operators}
The arithmetic operators perform arithmetic or bitwise logic between the operands. All operands to these operators are unsigned. If one operand is shorter than the other, it is zero-extended to match the length of the other.

\begin{itemize}
  \item \texttt{opA + opB}: Add \texttt{opA} and \texttt{opB}. The high bit of the result is a carry bit.
  \item \texttt{opA \textbf{and} opB}: Perform a bitwise AND between \texttt{opA} and \texttt{opB}.
  \item \texttt{opA \textbf{or} opB}: Perform a bitwise OR between \texttt{opA} and \texttt{opB}.
  \item \texttt{opA \textbf{xor} opB}: Perform a bitwise XOR between \texttt{opA} and \texttt{opB}.
  \item \texttt{\textbf{not} op}: Perform a bitwise negation of \texttt{op}.
\end{itemize}

\subsubsection{Logical Operators}
The logical operators yield 1 if the condition is satisfied and 0 if it is not. If one operand is shorter than the other, it is zero-extended to match the length of the other.

\begin{itemize}
  \item \texttt{opA = opB}: Satisfied if \texttt{opA} equals \texttt{opB}.
  \item \texttt{opA <> opB}: Satisfied if \texttt{opA} does not equal \texttt{opB}.
\end{itemize}

\subsubsection{Functions}

\begin{itemize}
  \item \texttt{sign\_extend\_16(op)}: Perform a sign extension of \texttt{op} by replicating the high bit until the total length is 16 bits.
  \item \texttt{decode\_imm\_al(op)}: Calculate the immediate value of an arithmetic or logical instruction according to the following table.\linebreak
    \begin{tabular}{|c c|}
    \hline
    \texttt{op} & Result \\
    \hline
    0 & \texttt{0x0000} \\
    1 & \texttt{0x0001} \\
    2 & \texttt{0x8000} \\
    3 & \texttt{TBD} \\
    4 & \texttt{0x00FF} \\
    5 & \texttt{0xFF00} \\
    6 & \texttt{0x7FFF} \\
    7 & \texttt{0xFFFF} \\
    \hline
    \end{tabular}
  \item \texttt{decode\_imm\_sr(op)}: Calculate the immediate value of a shift or rotate instruction according to the following table.\linebreak
    \begin{tabular}{|c c|}
    \hline
    \texttt{op} & Result \\
    \hline
    0 & 8 \\
    1 & 1 \\
    2 & 2 \\
    3 & 3 \\
    4 & 4 \\
    5 & 5 \\
    6 & 6 \\
    7 & 7 \\
    \hline
    \end{tabular}
\end{itemize}

\pagebreak

\section{List of Instructions}
The following pages provide a detailed description of instructions, arranged in alphabetical order.

Executing any instruction with an encoding not present on the following pages has \unpredictable{} behavior.
\pagebreak

\import{insns/}{index.tex}

\section{List of Assembly Directives}
TBD
\pagebreak

\section{Function Calling Sequence}
TBD
\pagebreak

\end{document}
