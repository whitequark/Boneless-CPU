\def←{$\leftarrow$}

\newcommand{\insnref}[1]{\hyperref[insn:#1]{\texttt{#1}}}

\newenvironment{instruction}[2]{
  \subsection[#1 (#2)]{#1 \hfill #2}
  \label{insn:#1}
  \vspace{0.5cm}

  \newcommand{\mnemonic}{\texttt{#1}{}}

  \newcommand{\field}[1]{\par\textbf{##1:}\par}
  \newcommand{\fieldindent}[2]{\field{##1}\begin{adjustwidth}{10pt}{0pt}##2\end{adjustwidth}}

  \newenvironment{encoding}[1][Encoding]{
    \newcommand{\bits}[2]{\multicolumn{####1}{c|}{####2}}
    \newcommand{\op}[2]{\bits{####1}{\texttt{####2}}}
    \newcommand{\opx}[2]{\bits{####1}{####2}}
    \newcommand{\reg}[1]{\bits{3}{R####1}}
    \newcommand{\imm}[1]{\bits{####1}{imm####1}}
    \newcommand{\off}[1]{\bits{####1}{off####1}}
    \newcommand{\exti}{\texttt{EXTI} & \op{3}{110} & \bits{13}{ext13} \\ \cline{2-17}}

    \field{##1}
    \begin{adjustwidth}{10pt}{0pt}
    \begin{tabular}{R{2cm}|*{16}{c|}}
    \cline{2-17}
    & F & E & D & C & B & A & 9 & 8 & 7 & 6 & 5 & 4 & 3 & 2 & 1 & 0 \\
    \cline{2-17}
  }{
    \cline{2-17}
    \end{tabular}
    \end{adjustwidth}
  }
  \newenvironment{encoding*}[1]{
    \begin{encoding}[Encoding (##1 form)]
  }{
    \end{encoding}
  }

  \newcommand{\assembly}[1]{\fieldindent{Assembly}{\texttt{##1}}}

  \newcommand{\purpose}[1]{\fieldindent{Purpose}{##1}}

  % Restrictions often include \unpredictable, which tends to mess with line break algorithm,
  % so we turn off hyphenation for these paragraphs with \raggedright.
  \newcommand{\restrictions}[1]{\fieldindent{Restrictions}{\raggedright##1}}

  \newenvironment{operation}{
    \newcommand{\K}[1]{\textbf{####1}}

    \newcommand{\aluRR}[1]{\begin{alltt}
    opA ← mem[W+Ra]\\
    opB ← mem[W+Rb]\\
    res ← ####1
    \end{alltt}}

    \newcommand{\aluRI}[2]{
    \begin{alltt}
    opA ← mem[W+Ra]\\
    \K{if} (has\_ext13)\\
    \K{then} opB ← \{ext13, imm3\}\\
    \K{else} opB ← decode\_imm\_####1(imm3)\\
    res ← ####2
    \end{alltt}}

    \newcommand{\wb}{mem[W+Rd] ← res[15:0]}

    \newcommand{\flagZS}{\begin{alltt}
    Z ← res[15:0] = 0\\
    S ← res[15]\\
    C ← \undefined\\
    V ← \undefined
    \end{alltt}}

    \newcommand{\flagZSCV}{\begin{alltt}
    Z ← res[15:0] = 0\\
    S ← res[15]\\
    C ← res[16]\\
    V ← (opA[15] = opB[15]) \K{and} (opA[15] <> res[15])
    \end{alltt}}

    \newcommand{\flagZSBV}{\begin{alltt}
    Z ← res[15:0] = 0\\
    S ← res[15]\\
    C ← res[16]\\
    V ← (opA[15] = \K{not} opB[15]) \K{and} (opA[15] <> res[15])
    \end{alltt}}

    \newcommand{\imm}[1]{\begin{alltt}
    \K{if} (has\_ext13)\\
    \K{then} imm ← \{ext13, imm{####1}[2:0]\}\\
    \K{else} imm ← sign\_extend\_16(imm{####1})
    \end{alltt}}

    \newcommand{\off}[1]{\begin{alltt}
    \K{if} (has\_ext13)\\
    \K{then} off ← \{ext13, off{####1}[2:0]\}\\
    \K{else} off ← sign\_extend\_16(off{####1})
    \end{alltt}}

    \newcommand{\jump}[1]{\begin{alltt}
    \K{if} (####1)\\
    \K{then} PC ← (PC + 1 + off)[15:0]\\
    \K{else} PC ← (PC + 1)[15:0]
    \end{alltt}}

    \field{Operation}
    \begin{adjustwidth}{10pt}{0pt}
    \begin{alltt}%
  }{%
    \end{alltt}
    \end{adjustwidth}
  }

  \newenvironment{remarks}{
    \field{Remarks}
    \begin{adjustwidth}{10pt}{0pt}
  }{
    \end{adjustwidth}
  }
}{
  \pagebreak
}

\begin{instruction}{ADC}{Add Register with Carry}
  \begin{encoding}
    \mnemonic & \op{5}{00010} & \reg{d} & \reg{a} & \op{2}{01} & \reg{b} \\
  \end{encoding}
  \assembly{\mnemonic{} Rd, Ra, Rb}
  \purpose{To add 16-bit integers in registers, with carry input.}
  \restrictions{None.}
  \begin{operation}\aluRR{opA + opB + C}\wb\flagZSCV\end{operation}
\begin{remarks}
A 32-bit addition with both operands in registers can be performed as follows:
\begin{alltt}
; Perform \string{R1, R0\string} ← \string{R3, R2\string} + \string{R5, R4\string}
    ADD  R0, R2, R4
    ADC  R1, R3, R5
\end{alltt}
\end{remarks}
\end{instruction}

\begin{instruction}{ADCI}{Add Immediate with Carry}
  \begin{encoding*}{short}
    \mnemonic & \op{5}{00011} & \reg{d} & \reg{a} & \op{2}{01} & \imm{3} \\
  \end{encoding*}
  \begin{encoding*}{long}
    \exti
    \mnemonic & \op{5}{00011} & \reg{d} & \reg{a} & \op{2}{01} & \imm{3} \\
  \end{encoding*}
  \assembly{\mnemonic{} Rd, Ra, imm}
  \purpose{To add a constant to a 16-bit integer in a register, with carry input.}
  \restrictions{None.}
  \begin{operation}\aluRI{al}{opA + opB + C}\wb\flagZSCV\end{operation}
\begin{remarks}
A 32-bit addition with a register and an immediate operand can be performed as follows:
\begin{alltt}
; Perform (R1|R0) ← (R3|R2) + 0x40001
    ADDI R0, R2, 1
    ADCI R1, R3, 4
\end{alltt}
\end{remarks}
\end{instruction}

\begin{instruction}{ADD}{Add Register}
  \begin{encoding}
    \mnemonic & \op{5}{00010} & \reg{d} & \reg{a} & \op{2}{00} & \reg{b} \\
  \end{encoding}
  \assembly{\mnemonic{} Rd, Ra, Rb}
  \purpose{To add 16-bit integers in registers.}
  \restrictions{None.}
  \begin{operation}\aluRR{opA + opB}\wb\flagZSCV\end{operation}
\end{instruction}

\begin{instruction}{ADDI}{Add Immediate}
  \begin{encoding*}{short}
    \mnemonic & \op{5}{00011} & \reg{d} & \reg{a} & \op{2}{00} & \imm{3} \\
  \end{encoding*}
  \begin{encoding*}{long}
    \exti
    \mnemonic & \op{5}{00011} & \reg{d} & \reg{a} & \op{2}{00} & \imm{3} \\
  \end{encoding*}
  \assembly{\mnemonic{} Rd, Ra, imm}
  \purpose{To add a constant to a 16-bit integer in a register.}
  \restrictions{None.}
  \begin{operation}\aluRI{al}{opA + opB}\wb\flagZSCV\end{operation}
\end{instruction}

\begin{instruction}{ADJW}{Adjust Window Address}
  \begin{encoding*}{short}
    \mnemonic & \op{5}{10100} & \opx{3}{000} & \op{3}{010} & \imm{5} \\
  \end{encoding*}
  \begin{encoding*}{long}
    \exti
    \mnemonic & \op{5}{10100} & \opx{3}{000} & \op{3}{010} & \imm{5} \\
  \end{encoding*}
  \assembly{\mnemonic{} imm}
  \purpose{To increase or decrease the address of the register window.}
  \restrictions{If \texttt{imm} contains a value that is not a multiple of 8, the behavior is \unpredictable. If the long form is used, and \texttt{imm5[4:3]} are non-zero, the behavior is \unpredictable.}
\begin{operation}\imm{5}
W ← (W + imm)[15:0]
\end{operation}
  \begin{remarks}This instruction may be used in a function prologue or epilogue.\end{remarks}
  \begin{notice}The interpretation of the immediate field of this instruction is not final.\end{notice}
\end{instruction}

\begin{instruction}{AND}{Bitwise AND with Register}
  \begin{encoding}
    \mnemonic & \op{5}{00000} & \reg{d} & \reg{a} & \op{2}{00} & \reg{b} \\
  \end{encoding}
  \assembly{\mnemonic{} Rd, Ra, Rb}
  \purpose{To perform bitwise AND between 16-bit integers in registers.}
  \restrictions{None.}
  \begin{operation}\aluRR{opA \K{and} opB}\wb\flagZS\end{operation}
\end{instruction}

\begin{instruction}{ANDI}{Bitwise AND with Immediate}
  \begin{encoding*}{short}
    \mnemonic & \op{5}{00001} & \reg{d} & \reg{a} & \op{2}{00} & \imm{3} \\
  \end{encoding*}
  \begin{encoding*}{long}
    \exti
    \mnemonic & \op{5}{00001} & \reg{d} & \reg{a} & \op{2}{00} & \imm{3} \\
  \end{encoding*}
  \assembly{\mnemonic{} Rd, Ra, imm}
  \purpose{To perform bitwise AND between a 16-bit integer in a register and a constant.}
  \restrictions{None.}
  \begin{operation}\aluRI{al}{opA \K{and} opB}\wb\flagZS\end{operation}
\end{instruction}

\begin{instruction}{CMP}{Compare to Register}
  \begin{encoding}
    \mnemonic & \op{5}{00000} & \op{3}{000} & \reg{a} & \op{2}{11} & \reg{b} \\
  \end{encoding}
  \assembly{\mnemonic{} Rd, Ra, Rb}
  \purpose{To compare 16-bit integers in registers.}
  \restrictions{None.}
  \begin{operation}\aluRR{opA - opB}\flagZSBV\end{operation}
  \begin{remarks}This instruction is identical to \texttt{SUB}, with the exception that it discards the computed value.\end{remarks}
\end{instruction}

\begin{instruction}{CMPI}{Compare to Immediate}
  \begin{encoding*}{short}
    \mnemonic & \op{5}{00001} & \op{3}{000} & \reg{a} & \op{2}{11} & \imm{3} \\
  \end{encoding*}
  \begin{encoding*}{long}
    \exti
    \mnemonic & \op{5}{00001} & \op{3}{000} & \reg{a} & \op{2}{11} & \imm{3} \\
  \end{encoding*}
  \assembly{\mnemonic{} Rd, Ra, imm}
  \purpose{To compare a constant to a 16-bit integer in a register.}
  \restrictions{None.}
  \begin{operation}\aluRI{opA - opB}\flagZSBV\end{operation}
  \begin{remarks}This instruction is identical to \texttt{SUBI}, with the exception that it discards the computed value.\end{remarks}
\end{instruction}

\begin{instruction}{EXTI}{Extend Immediate}
  \begin{encoding}
    \mnemonic & \op{3}{110} & \imm{13} \\
  \end{encoding}
  \assembly{\mnemonic{} imm}
  \purpose{To extend the range of immediate in the following instruction.}
  \restrictions{None.}
\begin{operation}
ext13 ← imm13
has\_ext13 ← 1
\end{operation}
  \begin{remarks}This instruction is automatically emitted by the assembler while translating other instructions. As it changes both the meaning of and the constraints placed on the immediate field in the following instruction, placing it manually may lead to unexpected results.\end{remarks}
\end{instruction}

\begin{instruction}{J}{Jump}
  \begin{encoding*}{short}
    \mnemonic & \op{4}{1011} & \op{4}{1111} & \off{8} \\
  \end{encoding*}
  \begin{encoding*}{long}
    \exti
    \mnemonic & \op{4}{1011} & \op{4}{1111} & \off{8} \\
  \end{encoding*}
  \assembly{\mnemonic{} label}
  \purpose{To unconditionally transfer control.}
  \restrictions{If the long form is used, and \texttt{off8[7:3]} are non-zero, the behavior is \unpredictable.}

  \begin{operation}\off{8}PC ← PC + 1 + off\end{operation}
\end{instruction}

\begin{instruction}{JAL}{Jump and Link}
  \begin{encoding*}{short}
    \mnemonic & \op{5}{10101} & \reg{d} & \off{8} \\
  \end{encoding*}
  \begin{encoding*}{long}
    \exti
    \mnemonic & \op{5}{10101} & \reg{d} & \off{8} \\
  \end{encoding*}
  \assembly{\mnemonic{} label}
  \purpose{To transfer control to a subroutine.}
  \restrictions{If the long form is used, and \texttt{off8[7:3]} are non-zero, the behavior is \unpredictable.}

\begin{operation}\off{8}
mem[W|Rd] ← PC + 1
PC ← PC + 1 + off
\end{operation}
\end{instruction}

\begin{instruction}{JC}{Jump if Carry}
  \begin{encoding*}{short}
    \mnemonic & \op{4}{1011} & \op{4}{1010} & \off{8} \\
  \end{encoding*}
  \begin{encoding*}{long}
    \exti
    \mnemonic & \op{4}{1011} & \op{4}{1010} & \off{8} \\
  \end{encoding*}
  \assembly{\mnemonic{} label}
  \purpose{To transfer control if an arithmetic operation resulted in unsigned overflow.}
  \restrictions{If the long form is used, and \texttt{off8[7:3]} are non-zero, the behavior is \unpredictable.}

  \begin{operation}\off{8}\jump{C}\end{operation}
  \begin{remarks}This instruction has the same encoding as \insnref{JUGE}.\end{remarks}
\end{instruction}

\begin{instruction}{JE}{Jump if Equal}
  \begin{encoding*}{short}
    \mnemonic & \op{3}{101} & \op{1}{1} & \op{4}{1000} & \off{8} \\
  \end{encoding*}
  \begin{encoding*}{long}
    \exti
    \mnemonic & \op{3}{101} & \op{1}{1} & \op{4}{1000} & \off{8} \\
  \end{encoding*}
  \assembly{\mnemonic{} label}
  \purpose{To transfer control after a \texttt{\insnref{CMP} Ra, Rb} instruction if \texttt{Ra} is equal to \texttt{Rb}.}
  \input{jump-restrictions.tex}
  \begin{operation}\off{8}\jump{Z}\end{operation}
  \begin{remarks}This instruction has the same encoding as \insnref{JZ}.\end{remarks}
\end{instruction}
 % alias
\begin{instruction}{JN}{Jump Never}
  \begin{encoding*}{short}
    \mnemonic & \op{3}{101} & \op{1}{1} & \op{4}{0111} & \off{8} \\
  \end{encoding*}
  \begin{encoding*}{long}
    \exti
    \mnemonic & \op{3}{101} & \op{1}{1} & \op{4}{0111} & \off{8} \\
  \end{encoding*}
  \assembly{\mnemonic{} label}
  \purpose{To serve as a placeholder for a jump instruction.}
  \input{jump-restrictions.tex}
  \begin{operation}PC ← PC + 1\end{operation}
  \begin{remarks}The \texttt{JN} instruction has no effect. It may be used as a placeholder for a different jump instruction with a predefiend offset when the exact condition is unknown, such as in certain self-modifying code.\end{remarks}
\end{instruction}

\begin{instruction}{JNC}{Jump if Not Carry}
  \begin{encoding*}{short}
    \mnemonic & \op{4}{1011} & \op{4}{0010} & \off{8} \\
  \end{encoding*}
  \begin{encoding*}{long}
    \exti
    \mnemonic & \op{4}{1011} & \op{4}{0010} & \off{8} \\
  \end{encoding*}
  \assembly{\mnemonic{} label}
  \purpose{To transfer control if the carry (C) flag is clear (e.g. the last arithmetic operation did not result in unsigned overflow).}
  \restrictions{If the long form is used, and \texttt{off8[7:3]} are non-zero, the behavior is \unpredictable.}

  \begin{operation}\off{8}\jump{\K{not} C}\end{operation}
  \begin{remarks}This instruction has the same encoding as \insnref{JULT}.\end{remarks}
\end{instruction}

\begin{instruction}{JNE}{Jump if Not Equal}
  \begin{encoding*}{short}
    \mnemonic & \op{4}{1011} & \op{4}{0000} & \off{8} \\
  \end{encoding*}
  \begin{encoding*}{long}
    \exti
    \mnemonic & \op{4}{1011} & \op{4}{0000} & \off{8} \\
  \end{encoding*}
  \assembly{\mnemonic{} label}
  \purpose{To transfer control after a \texttt{\insnref{CMP} Ra, Rb} instruction if \texttt{Ra} is not equal to \texttt{Rb}.}
  \restrictions{If the long form is used, and \texttt{off8[7:3]} are non-zero, the behavior is \unpredictable.}

  \begin{operation}\off{8}\jump{\K{not} Z}\end{operation}
  \begin{remarks}This instruction has the same encoding as \insnref{JNZ}.\end{remarks}
\end{instruction}
 % alias
\begin{instruction}{JNO}{Jump if Not Overflow}
  \begin{encoding*}{short}
    \mnemonic & \op{3}{101} & \op{1}{1} & \op{4}{0011} & \off{8} \\
  \end{encoding*}
  \begin{encoding*}{long}
    \exti
    \mnemonic & \op{3}{101} & \op{1}{1} & \op{4}{0011} & \off{8} \\
  \end{encoding*}
  \assembly{\mnemonic{} label}
  \purpose{To transfer control if an arithmetic or shift operation did not result in signed overflow.}
  \input{jump-restrictions.tex}
  \begin{operation}\off{8}\jump{\K{not} V}\end{operation}
\end{instruction}

\begin{instruction}{JNS}{Jump if Not Sign}
  \begin{encoding*}{short}
    \mnemonic & \op{3}{101} & \op{1}{1} & \op{4}{0001} & \off{8} \\
  \end{encoding*}
  \begin{encoding*}{long}
    \exti
    \mnemonic & \op{3}{101} & \op{1}{1} & \op{4}{0001} & \off{8} \\
  \end{encoding*}
  \assembly{\mnemonic{} label}
  \purpose{To transfer control if an arithmetic or shift operation produced a non-negative result.}
  \input{jump-restrictions.tex}
  \begin{operation}\off{8}\jump{\K{not} S}\end{operation}
\end{instruction}

\begin{instruction}{JNZ}{Jump if Not Zero}
  \begin{encoding*}{short}
    \mnemonic & \op{4}{1011} & \op{4}{0000} & \off{8} \\
  \end{encoding*}
  \begin{encoding*}{long}
    \exti
    \mnemonic & \op{4}{1011} & \op{4}{0000} & \off{8} \\
  \end{encoding*}
  \assembly{\mnemonic{} label}
  \purpose{To transfer control if an arithmetic or shift operation produced a non-zero result.}
  \restrictions{If the long form is used, and \texttt{off8[7:3]} are non-zero, the behavior is \unpredictable.}

  \begin{operation}\off{8}\jump{\K{not} Z}\end{operation}
  \begin{remarks}This instruction has the same encoding as \insnref{JNE}.\end{remarks}
\end{instruction}

\begin{instruction}{JO}{Jump if Overflow}
  \begin{encoding*}{short}
    \mnemonic & \op{4}{1011} & \op{4}{1011} & \off{8} \\
  \end{encoding*}
  \begin{encoding*}{long}
    \exti
    \mnemonic & \op{4}{1011} & \op{4}{1011} & \off{8} \\
  \end{encoding*}
  \assembly{\mnemonic{} label}
  \purpose{To transfer control if an arithmetic operation resulted in signed overflow.}
  \restrictions{If the long form is used, and \texttt{off8[7:3]} are non-zero, the behavior is \unpredictable.}

  \begin{operation}\off{8}\jump{V}\end{operation}
\end{instruction}

\begin{instruction}{JR}{Jump to Register}
  \begin{encoding*}{short}
    \mnemonic & \op{5}{10100} & \reg{s} & \op{3}{100} & \off{5} \\
  \end{encoding*}
  \begin{encoding*}{long}
    \exti
    \mnemonic & \op{5}{10100} & \reg{s} & \op{3}{100} & \off{5} \\
  \end{encoding*}
  \assembly{\mnemonic{} Rs, off}
  \purpose{To transfer control to a variable absolute address contained in a register, with a constant offset.}
  \restrictions{If the long form is used, and \texttt{off5[4:3]} are non-zero, the behavior is \unpredictable.}

\begin{operation}\off{5}
PC ← mem[W|Ra] + off
\end{operation}
\end{instruction}

\begin{instruction}{JRAL}{Jump to Register and Link}
  \begin{encoding}
    \mnemonic & \op{5}{10100} & \reg{d} & \op{3}{101} & \opx{2}{00} & \reg{b} \\
  \end{encoding}
  \assembly{\mnemonic{} Rd, Rb}
  \purpose{To transfer control to a subroutine whose variable absolute address is contained in a register.}
  \restrictions{None.}
\begin{operation}
addr ← mem[W|Rb]
mem[W|Rd] ← PC + 1
PC ← addr
\end{operation}
\end{instruction}

\begin{instruction}{JS}{Jump if Sign}
  \begin{encoding*}{short}
    \mnemonic & \op{3}{101} & \op{1}{1} & \op{4}{1001} & \off{8} \\
  \end{encoding*}
  \begin{encoding*}{long}
    \exti
    \mnemonic & \op{3}{101} & \op{1}{1} & \op{4}{1001} & \off{8} \\
  \end{encoding*}
  \assembly{\mnemonic{} label}
  \purpose{To transfer control if an arithmetic or shift operation produced a negative result.}
  \input{jump-restrictions.tex}
  \begin{operation}\off{8}\jump{S}\end{operation}
\end{instruction}

\begin{instruction}{JSGE}{Jump if Signed Greater or Equal}
  \begin{encoding*}{short}
    \mnemonic & \op{4}{1011} & \op{4}{0101} & \off{8} \\
  \end{encoding*}
  \begin{encoding*}{long}
    \exti
    \mnemonic & \op{4}{1011} & \op{4}{0101} & \off{8} \\
  \end{encoding*}
  \assembly{\mnemonic{} label}
  \purpose{To transfer control after a \texttt{\insnref{CMP} Ra, Rb} instruction if \texttt{Ra} is greater than or equal to \texttt{Rb} when interpreted as signed integer.}
  \restrictions{If the long form is used, and \texttt{off8[7:3]} are non-zero, the behavior is \unpredictable.}

  \begin{operation}\off{8}\jump{\K{not} (S \K{xor} V)}\end{operation}
\end{instruction}

\begin{instruction}{JSGT}{Jump if Signed Greater Than}
  \begin{encoding*}{short}
    \mnemonic & \op{3}{101} & \op{1}{1} & \op{4}{0110} & \off{8} \\
  \end{encoding*}
  \begin{encoding*}{long}
    \exti
    \mnemonic & \op{3}{101} & \op{1}{1} & \op{4}{0110} & \off{8} \\
  \end{encoding*}
  \assembly{\mnemonic{} label}
  \purpose{To transfer control after a \texttt{\insnref{CMP} Ra, Rb} instruction if \texttt{Ra} is greater than to \texttt{Rb} when interpreted as signed integer.}
  \input{jump-restrictions.tex}
  \begin{operation}\off{8}\jump{\K{not} ((S \K{xor} V) \K{or} Z)}\end{operation}
\end{instruction}

\input{JSLE.tex}
\begin{instruction}{JSLT}{Jump if Signed Less Than}
  \begin{encoding*}{short}
    \mnemonic & \op{4}{1011} & \op{4}{1101} & \off{8} \\
  \end{encoding*}
  \begin{encoding*}{long}
    \exti
    \mnemonic & \op{4}{1011} & \op{4}{1101} & \off{8} \\
  \end{encoding*}
  \assembly{\mnemonic{} label}
  \purpose{To transfer control after a \texttt{\insnref{CMP} Ra, Rb} instruction if \texttt{Ra} is less than \texttt{Rb} when interpreted as signed integer.}
  \restrictions{If the long form is used, and \texttt{off8[7:3]} are non-zero, the behavior is \unpredictable.}

  \begin{operation}\off{8}\jump{(S \K{xor} V)}\end{operation}
\end{instruction}

\begin{instruction}{JST}{Jump through Switch Table}
  \begin{encoding*}{short}
    \mnemonic & \op{5}{10100} & \reg{s} & \op{3}{111} & \off{5} \\
  \end{encoding*}
  \begin{encoding*}{long}
    \exti
    \mnemonic & \op{5}{10100} & \reg{s} & \op{3}{111} & \off{5} \\
  \end{encoding*}
  \assembly{\mnemonic{} Rs, off}
  \purpose{To transfer control to an address contained in a jump table at a variable offset, where the address is relative to the location of the table.}
  \restrictions{If the long form is used, and \texttt{off5[4:3]} are non-zero, the behavior is \unpredictable.}

\begin{operation}\off{5}
table ← PC + 1 + off
entry ← mem[W+Rs]
addr ← mem[table + entry]
PC ← (table + addr)[15:0]
\end{operation}
\end{instruction}

\begin{instruction}{JUGE}{Jump if Unsigned Greater or Equal}
  \begin{encoding*}{short}
    \mnemonic & \op{4}{1011} & \op{4}{1010} & \off{8} \\
  \end{encoding*}
  \begin{encoding*}{long}
    \exti
    \mnemonic & \op{4}{1011} & \op{4}{1010} & \off{8} \\
  \end{encoding*}
  \assembly{\mnemonic{} label}
  \purpose{To transfer control after a \texttt{\insnref{CMP} Ra, Rb} instruction if \texttt{Ra} is greater than or equal to \texttt{Rb} when interpreted as unsigned integers.}
  \restrictions{If the long form is used, and \texttt{off8[7:3]} are non-zero, the behavior is \unpredictable.}

  \begin{operation}\off{8}\jump{C}\end{operation}
  \begin{remarks}This instruction has the same encoding as \insnref{JC}.\end{remarks}
\end{instruction}
 % alias
\input{JUGT.tex}
\begin{instruction}{JULE}{Jump if Unsigned Less or Equal}
  \begin{encoding*}{short}
    \mnemonic & \op{3}{101} & \op{1}{1} & \op{4}{1110} & \off{8} \\
  \end{encoding*}
  \begin{encoding*}{long}
    \exti
    \mnemonic & \op{3}{101} & \op{1}{1} & \op{4}{1110} & \off{8} \\
  \end{encoding*}
  \assembly{\mnemonic{} label}
  \purpose{To transfer control after a \texttt{\insnref{CMP} Ra, Rb} instruction if \texttt{Ra} is less than or equal to \texttt{Rb} when interpreted as unsigned integer.}
  \input{jump-restrictions.tex}
  \begin{operation}\off{8}\jump{(\K{not} C) \K{or} V}\end{operation}
\end{instruction}

\begin{instruction}{JULT}{Jump if Unsigned Less Than}
  \begin{encoding*}{short}
    \mnemonic & \op{4}{1011} & \op{4}{0010} & \off{8} \\
  \end{encoding*}
  \begin{encoding*}{long}
    \exti
    \mnemonic & \op{4}{1011} & \op{4}{0010} & \off{8} \\
  \end{encoding*}
  \assembly{\mnemonic{} label}
  \purpose{To transfer control after a \texttt{\insnref{CMP} Ra, Rb} instruction if \texttt{Ra} is less than \texttt{Rb} when interpreted as unsigned integer.}
  \input{jump-restrictions.tex}
  \begin{operation}\off{8}\jump{\K{not} C}\end{operation}
  \begin{remarks}This instruction has the same encoding as \insnref{JNC}.\end{remarks}
\end{instruction}
 % alias
\begin{instruction}{JVT}{Jump through Virtual Table}
  \begin{encoding*}{short}
    \mnemonic & \op{5}{10100} & \reg{s} & \op{3}{110} & \off{5} \\
  \end{encoding*}
  \begin{encoding*}{long}
    \exti
    \mnemonic & \op{5}{10100} & \reg{s} & \op{3}{110} & \off{5} \\
  \end{encoding*}
  \assembly{\mnemonic{} Rs, off}
  \purpose{To transfer control to an address contained in a jump table at a constant offset, where the address is relative to the location of the table.}
  \restrictions{If the long form is used, and \texttt{off5[4:3]} are non-zero, the behavior is \unpredictable.}

\begin{operation}\off{5}
table ← mem[W+Rs]
addr ← mem[table + off]
PC ← table + addr
\end{operation}
\end{instruction}

\begin{instruction}{JZ}{Jump if Zero}
  \begin{encoding*}{short}
    \mnemonic & \op{4}{1011} & \op{4}{1000} & \off{8} \\
  \end{encoding*}
  \begin{encoding*}{long}
    \exti
    \mnemonic & \op{4}{1011} & \op{4}{1000} & \off{8} \\
  \end{encoding*}
  \assembly{\mnemonic{} label}
  \purpose{To transfer control if the last operation produced a result equal to zero.}
  \restrictions{If the long form is used, and \texttt{off8[7:3]} are non-zero, the behavior is \unpredictable.}

  \begin{operation}\off{8}\jump{Z}\end{operation}
  \begin{remarks}This instruction has the same encoding as \insnref{JE}.\end{remarks}
\end{instruction}

\begin{instruction}{LD}{Load with Address in Register}
  \begin{encoding*}{short}
    \mnemonic & \op{3}{010} & \op{2}{00} & \reg{d} & \reg{a} & \off{5} \\
  \end{encoding*}
  \begin{encoding*}{long}
    \exti
    \mnemonic & \op{3}{010} & \op{2}{00} & \reg{d} & \reg{a} & \off{5} \\
  \end{encoding*}
  \assembly{\mnemonic{} Rd, Ra, off}
  \purpose{To load a word from memory at an absolute address contained in a register, with a constant offset.}
  \restrictions{If the long form is used, and \texttt{off5[5:3]} are non-zero, the behavior is \unpredictable.}

\begin{operation}\off{5}
addr ← mem[W|Ra] + off
temp ← mem[addr]
mem[W|Rd] ← temp
\end{operation}
\end{instruction}

\begin{instruction}{LDR}{Load PC-relative}
  \begin{encoding*}{short}
    \mnemonic & \op{5}{01001} & \reg{d} & \reg{a} & \off{5} \\
  \end{encoding*}
  \begin{encoding*}{long}
    \exti
    \mnemonic & \op{5}{01001} & \reg{d} & \reg{a} & \off{5} \\
  \end{encoding*}
  \assembly{\mnemonic{} Rd, Ra, off}
  \purpose{To load a word from memory at a constant PC-relative address, with a variable offset.}
  \restrictions{If the long form is used, and \texttt{off5[4:3]} are non-zero, the behavior is \unpredictable.}

\begin{operation}\off{5}
addr ← PC + 1 + off + mem[W|Ra]
data ← mem[addr]
mem[W|Rd] ← data
\end{operation}
\end{instruction}

\begin{instruction}{LDW}{Adjust and Load Window Address}
  \begin{encoding*}{short}
    \mnemonic & \op{5}{10100} & \reg{d} & \op{3}{011} & \imm{5} \\
  \end{encoding*}
  \begin{encoding*}{long}
    \exti
    \mnemonic & \op{5}{10100} & \reg{d} & \op{3}{011} & \imm{5} \\
  \end{encoding*}
  \assembly{\mnemonic{} Rd, imm}
  \purpose{To increase or decrease the address of the register window, and retrieve the prior address of the register window.}
  \restrictions{If \texttt{imm} contains a value that is not a multiple of 8, the behavior is \unpredictable. If the long form is used, and \texttt{imm5[4:3]} are non-zero, the behavior is \unpredictable.}
\begin{operation}\imm{5}
old ← W
W ← (W + imm)[15:0]
mem[W+Rd] ← old
\end{operation}
  \begin{remarks}See also \insnref{STW}. This instruction may be used in a function prologue, where \texttt{Rd} is any register chosen to act as a frame pointer.\end{remarks}
  \begin{notice}The interpretation of the immediate field of this instruction is not final.\end{notice}
\end{instruction}

\begin{instruction}{LDX}{Load External}
  \begin{encoding*}{short}
    \mnemonic & \op{5}{01100} & \reg{d} & \reg{a} & \off{5} \\
  \end{encoding*}
  \begin{encoding*}{long}
    \exti
    \mnemonic & \op{5}{01100} & \reg{d} & \reg{a} & \off{5} \\
  \end{encoding*}
  \assembly{\mnemonic{} Rd, Ra, off}
  \purpose{To complete a load cycle on external bus at a variable address, with a constant offset.}
  \restrictions{If the long form is used, and \texttt{off5[4:3]} are non-zero, the behavior is \unpredictable.}

\begin{operation}\off{5}
addr ← mem[W+Ra] + off
data ← ext[addr]
mem[W+Rd] ← data
\end{operation}
\end{instruction}

\begin{instruction}{LDXA}{Load External Absolute}
  \begin{encoding*}{short}
    \mnemonic & \op{5}{01101} & \reg{d} & \off{8} \\
  \end{encoding*}
  \begin{encoding*}{long}
    \exti
    \mnemonic & \op{5}{01101} & \reg{d} & \off{8} \\
  \end{encoding*}
  \assembly{\mnemonic{} Rd, off}
  \purpose{To complete a load cycle on external bus at a constant absolute address.}
  \restrictions{If the long form is used, and \texttt{off8[7:3]} are non-zero, the behavior is \unpredictable.}

\begin{operation}\off{8}
data ← ext[off]
mem[W|Rd] ← data
\end{operation}
\end{instruction}

\begin{instruction}{MOV}{Move}
  \assembly{\mnemonic{} Rd, Rs}
  \purpose{To move a value from register to register.}
  \restrictions{None.}
  \begin{remarks}
The assembler does not translate any instructions for \texttt{\mnemonic} with identical \texttt{Rd} and \texttt{Rs}, and translates \texttt{\mnemonic} with any other register combination to
\begin{alltt}
  AND  Rd, Rs, Rs
\end{alltt}
  \end{remarks}
  \begin{notice}The exact translation of this mnemonic is not final.\end{notice}
\end{instruction}
 % pseudo
\begin{instruction}{MOVI}{Move Immediate}
  \begin{encoding*}{short}
    \mnemonic & \op{5}{10000} & \reg{d} & \imm{8} \\
  \end{encoding*}
  \begin{encoding*}{long}
    \exti
    \mnemonic & \op{5}{10000} & \reg{d} & \imm{8} \\
  \end{encoding*}
  \assembly{\mnemonic{} Rd, imm}
  \purpose{To load a register with a constant.}
  \restrictions{If the long form is used, and \texttt{imm8[8:3]} are non-zero, the behavior is \unpredictable.}
\begin{operation}\imm{8}
mem[W+Rd] ← imm
\end{operation}
\end{instruction}

\begin{instruction}{MOVR}{Move PC-relative Address}
  \begin{encoding*}{short}
    \mnemonic & \op{5}{10001} & \reg{d} & \off{8} \\
  \end{encoding*}
  \begin{encoding*}{long}
    \exti
    \mnemonic & \op{5}{10001} & \reg{d} & \off{8} \\
  \end{encoding*}
  \assembly{\mnemonic{} Rd, off}
  \purpose{To load a register with an address relative to PC with a constant offset.}
  \restrictions{If the long form is used, and \texttt{off8[7:3]} are non-zero, the behavior is \unpredictable.}

\begin{operation}\off{8}
mem[W+Rd] ← PC + 1 + off
\end{operation}
\end{instruction}

\begin{instruction}{OR}{Bitwise OR with Register}
  \begin{encoding}
    \mnemonic & \op{3}{000} & \op{2}{00} & \reg{d} & \reg{a} & \op{2}{01} & \reg{b} \\
  \end{encoding}
  \assembly{\mnemonic{} Rd, Ra, Rb}
  \purpose{To perform bitwise OR between 16-bit integers in registers.}
  \restrictions{None.}
  \begin{operation}\aluRR{opA \K{or} opB}\wb\flagZS\end{operation}
\end{instruction}

\begin{instruction}{ORI}{Bitwise OR with Immediate}
  \begin{encoding*}{short}
    \mnemonic & \op{5}{00001} & \reg{d} & \reg{a} & \op{2}{01} & \imm{3} \\
  \end{encoding*}
  \begin{encoding*}{long}
    \exti
    \mnemonic & \op{5}{00001} & \reg{d} & \reg{a} & \op{2}{01} & \imm{3} \\
  \end{encoding*}
  \assembly{\mnemonic{} Rd, Ra, imm}
  \purpose{To perform bitwise OR between a constant and a 16-bit integer in a register.}
  \restrictions{None.}
  \begin{operation}\aluRI{opA \K{or} opB}\wb\flagZS\end{operation}
\end{instruction}

\begin{instruction}{ROL}{Rotate Left}
  \begin{encoding}
    \mnemonic & \op{5}{00100} & \reg{d} & \reg{a} & \op{2}{01} & \reg{b} \\
  \end{encoding}
  \assembly{\mnemonic{} Rd, Ra, Rb}
  \purpose{To perform a left rotate of a 16-bit integer in a register by a variable bit amount.}
  \restrictions{If \texttt{Rb} contains a value greater than 15, the behavior is \unpredictable.}

  \begin{operation}\aluRR{\string{opA[15-opB:0], opA[15:15-opB]\string}}\wb\flagZS\end{operation}
\end{instruction}

\begin{instruction}{ROLI}{Rotate Left Immediate}
  \begin{encoding}
    \mnemonic & \op{5}{00101} & \reg{d} & \reg{a} & \op{2}{01} & \imm{3} \\
  \end{encoding}
  \assembly{\mnemonic{} Rd, Ra, amount}
  \purpose{To perform a left rotate of a 16-bit integer in a register by a constant bit amount.}
  \restrictions{The \texttt{amount} may be between 0 and 15, inclusive.}

  \begin{operation}\aluRI{sr}{\string{opA[15-opB:0], opA[15:15-opB]\string}}\wb\flagZS\end{operation}
\end{instruction}

\begin{instruction}{RORI}{Rotate Right Immediate}
  \assembly{\mnemonic{} Rd, Ra, amount}
  \purpose{To perform a right rotate of a 16-bit integer in a register by a constant bit amount.}
  \restrictions{The \texttt{amount} may be between 0 and 15, inclusive.}
  \begin{remarks}
The assembler translates \texttt{\mnemonic} with \texttt{amount} of 0 to
\begin{alltt}
  MOV  Rd, Ra
\end{alltt}
and \texttt{\mnemonic} with any other \texttt{amount} to
\begin{alltt}
  ROTI Rd, Rd, (15 - amount)
\end{alltt}
  \end{remarks}
\end{instruction}
 % pseudo
\begin{instruction}{SBB}{Subtract Register with Borrow}
  \begin{encoding}
    \mnemonic & \op{5}{00010} & \reg{d} & \reg{a} & \op{2}{11} & \reg{b} \\
  \end{encoding}
  \assembly{\mnemonic{} Rd, Ra, Rb}
  \purpose{To subtract 16-bit integers in registers, with borrow input.}
  \restrictions{None.}
  \begin{operation}\aluRR{opA - opB - \K{not} C}\wb\flagZSBV\end{operation}
\begin{remarks}
A 32-bit subtraction with both operands in registers can be performed as follows:
\begin{alltt}
; Perform (R1|R0) ← (R3|R2) - (R5|R4)
    SUB  R0, R2, R4
    SBB  R1, R3, R5
\end{alltt}
\end{remarks}
\end{instruction}

\begin{instruction}{SBBI}{Subtract Immediate with Borrow}
  \begin{encoding*}{short}
    \mnemonic & \op{3}{001} & \op{2}{01} & \reg{d} & \reg{a} & \op{2}{11} & \imm{3} \\
  \end{encoding*}
  \begin{encoding*}{long}
    \exti
    \mnemonic & \op{3}{001} & \op{2}{01} & \reg{d} & \reg{a} & \op{2}{11} & \imm{3} \\
  \end{encoding*}
  \assembly{\mnemonic{} Rd, Ra, imm}
  \purpose{To subtract a constant from a 16-bit integer in a register, with borrow input.}
  \restrictions{None.}
  \begin{operation}\aluRI{opA - opB - \K{not} C}\wb\flagZSBV\end{operation}
  \begin{remarks}
  A 32-bit subtraction with a register and an immediate operand can be performed as follows:
  \begin{alltt}
  ; Perform (R1|R0) ← (R3|R2) - 0x40001
  SUBI R0, R2, 1
  SBBI R1, R3, 4
  \end{alltt}
  \end{remarks}
\end{instruction}

\begin{instruction}{SLL}{Shift Left Logical}
  \begin{encoding}
    \mnemonic & \op{3}{000} & \op{2}{10} & \reg{d} & \reg{a} & \op{2}{00} & \reg{b} \\
  \end{encoding}
  \assembly{\mnemonic{} Rd, Ra, Rb}
  \purpose{To perform a left logical shift of a 16-bit integer in a register by a variable bit amount.}
  \restrictions{If \texttt{Rb} contains a value greater than 15, the behavior is \unpredictable.}

  \begin{operation}\aluRR{opA[16-opB:0]|0\string{opB\string}}\wb\flagZS\end{operation}
\end{instruction}

\begin{instruction}{SLLI}{Shift Left Logical Immediate}
  \begin{encoding}
    \mnemonic & \op{5}{00101} & \reg{d} & \reg{a} & \op{2}{00} & \imm{3} \\
  \end{encoding}
  \assembly{\mnemonic{} Rd, Ra, amount}
  \purpose{To perform a left logical shift of a 16-bit integer in a register by a constant bit amount.}
  \restrictions{The \texttt{amount} may be between 0 and 15, inclusive.}

  \begin{operation}\aluRI{sr}{opA[15-imm3:0]|0\string{imm3+1\string}}\wb\flagZS\end{operation}
\end{instruction}

\begin{instruction}{SRA}{Shift Right Arithmetical}
  \begin{encoding}
    \mnemonic & \op{5}{00100} & \reg{d} & \reg{a} & \op{2}{11} & \reg{b} \\
  \end{encoding}
  \assembly{\mnemonic{} Rd, Ra, Rb}
  \purpose{To perform a right arithmetical shift of a 16-bit integer in a register by a variable bit amount.}
  \restrictions{If \texttt{Rb} contains a value greater than 15, the behavior is \unpredictable.}

  \begin{operation}\aluRR{opA[15]\string{opB\string}|opA[16:16-opB]}\wb\flagZS\end{operation}
\end{instruction}

\begin{instruction}{SRAI}{Shift Right Arithmetical Immediate}
  \begin{encoding}
    \mnemonic & \op{5}{00100} & \reg{d} & \reg{a} & \op{2}{11} & \imm{3} \\
  \end{encoding}
  \assembly{\mnemonic{} Rd, Ra, amount}
  \purpose{To perform a right arithmetical shift of a 16-bit integer in a register by a constant bit amount.}
  \restrictions{The \texttt{amount} may be between 0 and 15, inclusive.}
  \begin{operation}\aluRR{opA[15]\string{imm3+1\string}|opA[16:15-imm3]}\wb\flagZS\end{operation}
  \begin{remarks}
The instruction encoding allows directly representing any \texttt{amount} between 1 and 8, inclusive. The assembler translates \texttt{\mnemonic} with \texttt{amount} of 0 to
\begin{alltt}
  MOV  Rd, Ra
\end{alltt}
and \texttt{\mnemonic} with \texttt{amount} greater than 8 to
\begin{alltt}
  \mnemonic Rd, Ra, 8
  \mnemonic Rd, Rd, (amount - 8)
\end{alltt}
\end{remarks}

\end{instruction}

\begin{instruction}{SRL}{Shift Right Logical}
  \begin{encoding}
    \mnemonic & \op{5}{00100} & \reg{d} & \reg{a} & \op{2}{10} & \reg{b} \\
  \end{encoding}
  \assembly{\mnemonic{} Rd, Ra, Rb}
  \purpose{To perform a right logical shift of a 16-bit integer in a register by a variable bit amount.}
  \restrictions{If \texttt{Rb} contains a value greater than 15, the behavior is \unpredictable.}

  \begin{operation}\aluRR{\string{\string{opB\string{0\string}\string}, opA[15:opB]\string}}\wb\flagZS\end{operation}
\end{instruction}

\begin{instruction}{SRLI}{Shift Right Logical Immediate}
  \begin{encoding}
    \mnemonic & \op{5}{00101} & \reg{d} & \reg{a} & \op{2}{10} & \imm{3} \\
  \end{encoding}
  \assembly{\mnemonic{} Rd, Ra, amount}
  \purpose{To perform a right logical shift of a 16-bit integer in a register by a constant bit amount.}
  \restrictions{The \texttt{amount} may be between 0 and 15, inclusive.}

  \begin{operation}\aluRI{sr}{0\string{imm3+1\string}|opA[16:15-imm3]}\wb\flagZS\end{operation}
\end{instruction}

\begin{instruction}{ST}{Store}
  \begin{encoding*}{short}
    \mnemonic & \op{5}{01010} & \reg{s} & \reg{a} & \off{5} \\
  \end{encoding*}
  \begin{encoding*}{long}
    \exti
    \mnemonic & \op{5}{01010} & \reg{s} & \reg{a} & \off{5} \\
  \end{encoding*}
  \assembly{\mnemonic{} Rs, Ra, off}
  \purpose{To store a word to memory at a variable address, with a constant offset.}
  \restrictions{If the long form is used, and \texttt{off5[4:3]} are non-zero, the behavior is \unpredictable.}

\begin{operation}\off{5}
addr ← mem[W+Ra] + off
data ← mem[W+Rs]
mem[addr] ← data
\end{operation}
\end{instruction}

\begin{instruction}{STR}{Store PC-relative}
  \begin{encoding*}{short}
    \mnemonic & \op{5}{01011} & \reg{s} & \reg{a} & \off{5} \\
  \end{encoding*}
  \begin{encoding*}{long}
    \exti
    \mnemonic & \op{5}{01011} & \reg{s} & \reg{a} & \off{5} \\
  \end{encoding*}
  \assembly{\mnemonic{} Rs, Ra, off}
  \purpose{To store a word to memory at a constant PC-relative address, with a variable offset.}
  \restrictions{If the long form is used, and \texttt{off5[4:3]} are non-zero, the behavior is \unpredictable.}

\begin{operation}\off{5}
addr ← PC + 1 + off + mem[W+Ra]
data ← mem[W+Rs]
mem[addr] ← data
\end{operation}
\end{instruction}

\begin{instruction}{STW}{Store to Window Address}
  \begin{encoding}
    \mnemonic & \op{5}{10100} & \opx{3}{000} & \op{3}{000} & \opx{2}{00} & \reg{b} \\
  \end{encoding}
  \assembly{\mnemonic{} Rb}
  \purpose{To arbitrarily change the address of the register window.}
  \restrictions{If \texttt{Rb} contains a value that is not a multiple of 8, the behavior is \unpredictable.}
\begin{operation}
W ← mem[W+Rb]
\end{operation}
  \begin{remarks}See also \insnref{LDW}. This instruction may be used in a function epilogue, where \texttt{Rb} is any register chosen to act as a frame pointer.\end{remarks}
\end{instruction}

\begin{instruction}{STX}{Store External}
  \begin{encoding*}{short}
    \mnemonic & \op{5}{01110} & \reg{s} & \reg{a} & \off{5} \\
  \end{encoding*}
  \begin{encoding*}{long}
    \exti
    \mnemonic & \op{5}{01110} & \reg{s} & \reg{a} & \off{5} \\
  \end{encoding*}
  \assembly{\mnemonic{} Rs, Ra, off}
  \purpose{To complete a store cycle on external bus at a variable address, with a constant offset.}
  \input{mem5-restrictions.tex}
\begin{operation}\off{5}
addr ← mem[W|Ra] + off
temp ← mem[W|Rs]
ext[addr] ← temp
\end{operation}
\end{instruction}

\begin{instruction}{STXA}{Store External Absolute}
  \begin{encoding*}{short}
    \mnemonic & \op{5}{01111} & \reg{s} & \off{8} \\
  \end{encoding*}
  \begin{encoding*}{long}
    \exti
    \mnemonic & \op{5}{01111} & \reg{s} & \off{8} \\
  \end{encoding*}
  \assembly{\mnemonic{} Rs, off}
  \purpose{To complete a store cycle on external bus at a constant absolute address.}
  \restrictions{If the long form is used, and \texttt{off8[7:3]} are non-zero, the behavior is \unpredictable.}

\begin{operation}\off{8}
data ← mem[W|Rs]
ext[off] ← data
\end{operation}
\end{instruction}

\begin{instruction}{SUB}{Subtract Register}
  \begin{encoding}
    \mnemonic & \op{5}{00010} & \reg{d} & \reg{a} & \op{2}{10} & \reg{b} \\
  \end{encoding}
  \assembly{\mnemonic{} Rd, Ra, Rb}
  \purpose{To subtract 16-bit integers in registers.}
  \restrictions{None.}
  \begin{operation}\aluRR{opA - opB}\wb\flagZSBV\end{operation}
\end{instruction}

\begin{instruction}{SUBI}{Subtract Immediate}
  \begin{encoding*}{short}
    \mnemonic & \op{3}{001} & \op{2}{01} & \reg{d} & \reg{a} & \op{2}{10} & \imm{3} \\
  \end{encoding*}
  \begin{encoding*}{long}
    \exti
    \mnemonic & \op{3}{001} & \op{2}{01} & \reg{d} & \reg{a} & \op{2}{10} & \imm{3} \\
  \end{encoding*}
  \assembly{\mnemonic{} Rd, Ra, imm}
  \purpose{To subtract a constant from a 16-bit integer in a register.}
  \restrictions{None.}
  \begin{operation}\aluRI{opA - opB}\wb\flagZSBV\end{operation}
\end{instruction}

\begin{instruction}{XCHG}{Exchange Registers}
  \assembly{\mnemonic{} Ra, Rb}
  \purpose{To exchange the values of two registers.}
  \restrictions{None.}
  \begin{remarks}
The assembler does not translate any instructions for \texttt{\mnemonic} with identical \texttt{Ra} and \texttt{Rb}, and translates \texttt{\mnemonic} with any other register combination to
\begin{alltt}
  XOR  Ra, Ra, Rb
  XOR  Rb, Rb, Ra
  XOR  Ra, Ra, Rb
\end{alltt}
  \end{remarks}
\end{instruction}
 % pseudo
\begin{instruction}{XCHW}{Exchange Window Address}
  \begin{encoding}
    \mnemonic & \op{5}{10100} & \reg{d} & \op{3}{001} & \opx{2}{00} & \reg{b} \\
  \end{encoding}
  \assembly{\mnemonic{} Rd, Rb}
  \purpose{To exchange the address of the register window with a general purpose register.}
  \restrictions{If \texttt{Rb} contains a value that is not a multiple of 8, the behavior is \unpredictable.}
\begin{operation}
old ← W
W ← mem[W+Rb]
mem[W+Rd] ← old
\end{operation}
\begin{remarks}
This instruction may be used in a context switch routine. For example, if multiple register windows are set up such that each contains the address of the next one in \texttt{R7}, the following code may be used to switch contexts:
\begin{alltt}
yield:
    XCHW R7, R7
    JR   R0
; Elsewhere:
    JAL R0, yield
\end{alltt}
\end{remarks}
\end{instruction}

\begin{instruction}{XOR}{Bitwise XOR with Register}
  \begin{encoding}
    \mnemonic & \op{5}{00000} & \reg{d} & \reg{a} & \op{2}{10} & \reg{b} \\
  \end{encoding}
  \assembly{\mnemonic{} Rd, Ra, Rb}
  \purpose{To perform bitwise XOR between 16-bit integers in registers.}
  \restrictions{None.}
  \begin{operation}\aluRR{opA \K{xor} opB}\wb\flagZS\end{operation}
\end{instruction}

\begin{instruction}{XORI}{Bitwise XOR with Immediate}
  \begin{encoding*}{short}
    \mnemonic & \op{5}{00001} & \reg{d} & \reg{a} & \op{2}{10} & \imm{3} \\
  \end{encoding*}
  \begin{encoding*}{long}
    \exti
    \mnemonic & \op{5}{00001} & \reg{d} & \reg{a} & \op{2}{10} & \imm{3} \\
  \end{encoding*}
  \assembly{\mnemonic{} Rd, Ra, imm}
  \purpose{To perform bitwise XOR between a 16-bit integer in a register and a constant.}
  \restrictions{None.}
  \begin{operation}\aluRI{al}{opA \K{xor} opB}\wb\flagZS\end{operation}
\end{instruction}

