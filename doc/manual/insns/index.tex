\def←{$\leftarrow$}

\newcommand{\insnref}[1]{\hyperref[insn:#1]{\texttt{#1}}}

\newenvironment{instruction}[2]{
  \subsection[#1 (#2)]{#1 \hfill #2}
  \label{insn:#1}
  \vspace{0.5cm}

  \newcommand{\mnemonic}{\texttt{#1}{}}

  \newcommand{\field}[1]{\par\textbf{##1:}\par}
  \newcommand{\fieldindent}[2]{\field{##1}\begin{adjustwidth}{10pt}{0pt}##2\end{adjustwidth}}

  \newenvironment{encoding}[1][Encoding]{
    \newcommand{\bits}[2]{\multicolumn{####1}{c|}{####2}}
    \newcommand{\op}[2]{\bits{####1}{\texttt{####2}}}
    \newcommand{\opx}[2]{\bits{####1}{####2}}
    \newcommand{\reg}[1]{\bits{3}{R####1}}
    \newcommand{\imm}[1]{\bits{####1}{imm####1}}
    \newcommand{\off}[1]{\bits{####1}{off####1}}
    \newcommand{\exti}{\texttt{EXTI} & \op{3}{110} & \bits{13}{ext13} \\ \cline{2-17}}

    \field{##1}
    \begin{adjustwidth}{10pt}{0pt}
    \begin{tabular}{R{2cm}|*{16}{c|}}
    \cline{2-17}
    & F & E & D & C & B & A & 9 & 8 & 7 & 6 & 5 & 4 & 3 & 2 & 1 & 0 \\
    \cline{2-17}
  }{
    \cline{2-17}
    \end{tabular}
    \end{adjustwidth}
  }
  \newenvironment{encoding*}[1]{
    \begin{encoding}[Encoding (##1 form)]
  }{
    \end{encoding}
  }

  \newcommand{\assembly}[1]{\fieldindent{Assembly}{\texttt{##1}}}

  \newcommand{\purpose}[1]{\fieldindent{Purpose}{##1}}

  % Restrictions often include \unpredictable, which tends to mess with line break algorithm,
  % so we turn off hyphenation for these paragraphs with \raggedright.
  \newcommand{\restrictions}[1]{\fieldindent{Restrictions}{\raggedright##1}}

  \newenvironment{operation}{
    \newcommand{\K}[1]{\textbf{####1}}

    \newcommand{\aluRR}[1]{\begin{alltt}
    opA ← mem[W+Ra]\\
    opB ← mem[W+Rb]\\
    res ← ####1
    \end{alltt}}

    \newcommand{\aluRI}[2]{
    \begin{alltt}
    opA ← mem[W+Ra]\\
    \K{if} (has\_ext13)\\
    \K{then} opB ← \string{ext13, imm3\string}\\
    \K{else} opB ← decode\_imm\_####1(imm3)\\
    res ← ####2
    \end{alltt}}

    \newcommand{\wb}{mem[W+Rd] ← res}

    \newcommand{\flagZS}{\begin{alltt}
    Z ← res[15:0] = 0\\
    S ← res[15]\\
    C ← \undefined\\
    V ← \undefined
    \end{alltt}}

    \newcommand{\flagZSCV}{\begin{alltt}
    Z ← res[15:0] = 0\\
    S ← res[15]\\
    C ← res[16]\\
    V ← (opA[15] = opB[15]) \K{and} (opA[15] <> res[15])
    \end{alltt}}

    \newcommand{\flagZSBV}{\begin{alltt}
    Z ← res[15:0] = 0\\
    S ← res[15]\\
    C ← res[16]\\
    V ← (opA[15] = \K{not} opB[15]) \K{and} (opA[15] <> res[15])
    \end{alltt}}

    \newcommand{\imm}[1]{\begin{alltt}
    \K{if} (has\_ext13)\\
    \K{then} imm ← \string{ext13, imm{####1}[2:0]\string}\\
    \K{else} imm ← sign\_extend\_16(imm{####1})
    \end{alltt}}

    \newcommand{\off}[1]{\begin{alltt}
    \K{if} (has\_ext13)\\
    \K{then} off ← \string{ext13, off{####1}[2:0]\string}\\
    \K{else} off ← sign\_extend\_16(off{####1})
    \end{alltt}}

    \newcommand{\jump}[1]{\begin{alltt}
    \K{if} (####1)\\
    \K{then} PC ← PC + 1 + off\\
    \K{else} PC ← PC + 1
    \end{alltt}}

    \field{Operation}
    \begin{adjustwidth}{10pt}{0pt}
    \begin{alltt}%
  }{%
    \end{alltt}
    \end{adjustwidth}
  }

  \newenvironment{remarks}{
    \field{Remarks}
    \begin{adjustwidth}{10pt}{0pt}
  }{
    \end{adjustwidth}
  }
}{
  \pagebreak
}

\begin{instruction}{ADC}{Add Register with Carry}
  \begin{encoding}
    \mnemonic & \op{5}{00010} & \reg{d} & \reg{a} & \op{2}{01} & \reg{b} \\
  \end{encoding}
  \assembly{\mnemonic{} Rd, Ra, Rb}
  \purpose{To add 16-bit integers in registers, with carry input.}
  \restrictions{None.}
  \begin{operation}\aluRR{opA + opB + C}\wb\flagZSCV\end{operation}
\begin{remarks}
A 32-bit addition with both operands in registers can be performed as follows:
\begin{alltt}
; Perform \string{R1, R0\string} ← \string{R3, R2\string} + \string{R5, R4\string}
    ADD  R0, R2, R4
    ADC  R1, R3, R5
\end{alltt}
\end{remarks}
\end{instruction}

\begin{instruction}{ADCI}{Add Immediate with Carry}
  \begin{encoding*}{short}
    \mnemonic & \op{5}{00011} & \reg{d} & \reg{a} & \op{2}{01} & \imm{3} \\
  \end{encoding*}
  \begin{encoding*}{long}
    \exti
    \mnemonic & \op{5}{00011} & \reg{d} & \reg{a} & \op{2}{01} & \imm{3} \\
  \end{encoding*}
  \assembly{\mnemonic{} Rd, Ra, imm}
  \purpose{To add a constant to a 16-bit integer in a register, with carry input.}
  \restrictions{None.}
  \begin{operation}\aluRI{al}{opA + opB + C}\wb\flagZSCV\end{operation}
\begin{remarks}
A 32-bit addition with a register and an immediate operand can be performed as follows:
\begin{alltt}
; Perform \string{R1, R0\string} ← \string{R3, R2\string} + 0x40001
    ADDI R0, R2, 1
    ADCI R1, R3, 4
\end{alltt}
\end{remarks}
\end{instruction}

\input{ADD.tex}
\input{ADDI.tex}
\begin{instruction}{ADJW}{Adjust Window Address}
  \begin{encoding*}{short}
    \mnemonic & \op{5}{10100} & \opx{3}{000} & \op{3}{010} & \imm{5} \\
  \end{encoding*}
  \begin{encoding*}{long}
    \exti
    \mnemonic & \op{5}{10100} & \opx{3}{000} & \op{3}{010} & \imm{5} \\
  \end{encoding*}
  \assembly{\mnemonic{} imm}
  \purpose{To increase or decrease the address of the register window.}
  \restrictions{If \texttt{imm} contains a value that is not a multiple of 8, the behavior is \unpredictable. If the long form is used, and \texttt{imm5[4:3]} are non-zero, the behavior is \unpredictable.}
\begin{operation}\imm{5}
W ← W + imm
\end{operation}
  \begin{remarks}This instruction may be used in a function prologue or epilogue.\end{remarks}
  \begin{notice}The interpretation of the immediate field of this instruction is not final.\end{notice}
\end{instruction}

\input{AND.tex}
\input{ANDI.tex}
\begin{instruction}{BC}{Branch if Carry}
  \begin{encoding*}{short}
    \mnemonic & \op{4}{1011} & \op{4}{1010} & \off{8} \\
  \end{encoding*}
  \begin{encoding*}{long}
    \exti
    \mnemonic & \op{4}{1011} & \op{4}{1010} & \off{8} \\
  \end{encoding*}
  \assembly{\mnemonic{} label}
  \purpose{To transfer control if the last arithmetic operation resulted in unsigned overflow.}
  \input{off8-restrictions.tex}
  \begin{operation}\off{8}\jump{C}\end{operation}
  \begin{remarks}This instruction has the same encoding as \insnref{BC1}.\end{remarks}
\end{instruction}

\begin{instruction}{BC0}{Branch if Carry is 0}
  \begin{encoding*}{short}
    \mnemonic & \op{4}{1011} & \op{4}{0010} & \off{8} \\
  \end{encoding*}
  \begin{encoding*}{long}
    \exti
    \mnemonic & \op{4}{1011} & \op{4}{0010} & \off{8} \\
  \end{encoding*}
  \assembly{\mnemonic{} label}
  \purpose{To transfer control if the carry (C) flag is 0.}
  \input{off8-restrictions.tex}
  \begin{operation}\off{8}\jump{\K{not} C}\end{operation}
\end{instruction}

\begin{instruction}{BC1}{Branch if Carry is 1}
  \begin{encoding*}{short}
    \mnemonic & \op{4}{1011} & \op{4}{1010} & \off{8} \\
  \end{encoding*}
  \begin{encoding*}{long}
    \exti
    \mnemonic & \op{4}{1011} & \op{4}{1010} & \off{8} \\
  \end{encoding*}
  \assembly{\mnemonic{} label}
  \purpose{To transfer control if the carry (C) flag is 1.}
  \input{off8-restrictions.tex}
  \begin{operation}\off{8}\jump{C}\end{operation}
\end{instruction}

\begin{instruction}{BEQ}{Branch if Equal}
  \begin{encoding*}{short}
    \mnemonic & \op{4}{1011} & \op{4}{1000} & \off{8} \\
  \end{encoding*}
  \begin{encoding*}{long}
    \exti
    \mnemonic & \op{4}{1011} & \op{4}{1000} & \off{8} \\
  \end{encoding*}
  \assembly{\mnemonic{} label}
  \purpose{To transfer control after a \texttt{\insnref{CMP} Ra, Rb} instruction if \texttt{Ra} is equal to \texttt{Rb}.}
  \input{off8-restrictions.tex}
  \begin{operation}\off{8}\jump{Z}\end{operation}
  \begin{remarks}This instruction has the same encoding as \insnref{BZ1}.\end{remarks}
\end{instruction}
 % alias
\begin{instruction}{BGES}{Branch if Greater or Equal, Signed}
  \begin{encoding*}{short}
    \mnemonic & \op{4}{1011} & \op{4}{0101} & \off{8} \\
  \end{encoding*}
  \begin{encoding*}{long}
    \exti
    \mnemonic & \op{4}{1011} & \op{4}{0101} & \off{8} \\
  \end{encoding*}
  \assembly{\mnemonic{} label}
  \purpose{To transfer control after a \texttt{\insnref{CMP} Ra, Rb} instruction if \texttt{Ra} is greater than or equal to \texttt{Rb} when interpreted as signed integers.}
  \input{off8-restrictions.tex}
  \begin{operation}\off{8}\jump{\K{not} (S \K{xor} V)}\end{operation}
\end{instruction}

\begin{instruction}{BGEU}{Branch if Greater or Equal, Unsigned}
  \begin{encoding*}{short}
    \mnemonic & \op{4}{1011} & \op{4}{1010} & \off{8} \\
  \end{encoding*}
  \begin{encoding*}{long}
    \exti
    \mnemonic & \op{4}{1011} & \op{4}{1010} & \off{8} \\
  \end{encoding*}
  \assembly{\mnemonic{} label}
  \purpose{To transfer control after a \texttt{\insnref{CMP} Ra, Rb} instruction if \texttt{Ra} is greater than or equal to \texttt{Rb} when interpreted as unsigned integers.}
  \input{off8-restrictions.tex}
  \begin{operation}\off{8}\jump{C}\end{operation}
  \begin{remarks}This instruction has the same encoding as \insnref{BC1}.\end{remarks}
\end{instruction}
 % alias
\begin{instruction}{BGTS}{Branch if Greater Than, Signed}
  \begin{encoding*}{short}
    \mnemonic & \op{4}{1011} & \op{4}{0110} & \off{8} \\
  \end{encoding*}
  \begin{encoding*}{long}
    \exti
    \mnemonic & \op{4}{1011} & \op{4}{0110} & \off{8} \\
  \end{encoding*}
  \assembly{\mnemonic{} label}
  \purpose{To transfer control after a \texttt{\insnref{CMP} Ra, Rb} instruction if \texttt{Ra} is greater than \texttt{Rb} when interpreted as signed integers.}
  \input{off8-restrictions.tex}
  \begin{operation}\off{8}\jump{\K{not} ((S \K{xor} V) \K{or} Z)}\end{operation}
\end{instruction}

\begin{instruction}{BGTU}{Branch if Greater Than, Unsigned}
  \begin{encoding*}{short}
    \mnemonic & \op{4}{1011} & \op{4}{0100} & \off{8} \\
  \end{encoding*}
  \begin{encoding*}{long}
    \exti
    \mnemonic & \op{4}{1011} & \op{4}{0100} & \off{8} \\
  \end{encoding*}
  \assembly{\mnemonic{} label}
  \purpose{To transfer control after a \texttt{\insnref{CMP} Ra, Rb} instruction if \texttt{Ra} is greater than \texttt{Rb} when interpreted as unsigned integers.}
  \input{off8-restrictions.tex}
  \begin{operation}\off{8}\jump{\K{not} ((\K{not} C) \K{or} Z)}\end{operation}
\end{instruction}

\begin{instruction}{BLES}{Branch if Less or Equal, Signed}
  \begin{encoding*}{short}
    \mnemonic & \op{4}{1011} & \op{4}{1110} & \off{8} \\
  \end{encoding*}
  \begin{encoding*}{long}
    \exti
    \mnemonic & \op{4}{1011} & \op{4}{1110} & \off{8} \\
  \end{encoding*}
  \assembly{\mnemonic{} label}
  \purpose{To transfer control after a \texttt{\insnref{CMP} Ra, Rb} instruction if \texttt{Ra} is less than or equal to \texttt{Rb} when interpreted as signed integers.}
  \input{off8-restrictions.tex}
  \begin{operation}\off{8}\jump{(S \K{xor} V) \K{or} Z}\end{operation}
\end{instruction}

\begin{instruction}{BLEU}{Branch if Less or Equal, Unsigned}
  \begin{encoding*}{short}
    \mnemonic & \op{4}{1011} & \op{4}{1100} & \off{8} \\
  \end{encoding*}
  \begin{encoding*}{long}
    \exti
    \mnemonic & \op{4}{1011} & \op{4}{1100} & \off{8} \\
  \end{encoding*}
  \assembly{\mnemonic{} label}
  \purpose{To transfer control after a \texttt{\insnref{CMP} Ra, Rb} instruction if \texttt{Ra} is less than or equal to \texttt{Rb} when interpreted as unsigned integers.}
  \input{off8-restrictions.tex}
  \begin{operation}\off{8}\jump{(\K{not} C) \K{or} Z}\end{operation}
\end{instruction}

\begin{instruction}{BLTS}{Branch if Less Than, Signed}
  \begin{encoding*}{short}
    \mnemonic & \op{4}{1011} & \op{4}{1101} & \off{8} \\
  \end{encoding*}
  \begin{encoding*}{long}
    \exti
    \mnemonic & \op{4}{1011} & \op{4}{1101} & \off{8} \\
  \end{encoding*}
  \assembly{\mnemonic{} label}
  \purpose{To transfer control after a \texttt{\insnref{CMP} Ra, Rb} instruction if \texttt{Ra} is less than \texttt{Rb} when interpreted as signed integers.}
  \input{off8-restrictions.tex}
  \begin{operation}\off{8}\jump{S \K{xor} V}\end{operation}
\end{instruction}

\begin{instruction}{BLTU}{Branch if Less Than, Unsigned}
  \begin{encoding*}{short}
    \mnemonic & \op{4}{1011} & \op{4}{0010} & \off{8} \\
  \end{encoding*}
  \begin{encoding*}{long}
    \exti
    \mnemonic & \op{4}{1011} & \op{4}{0010} & \off{8} \\
  \end{encoding*}
  \assembly{\mnemonic{} label}
  \purpose{To transfer control after a \texttt{\insnref{CMP} Ra, Rb} instruction if \texttt{Ra} is less than \texttt{Rb} when interpreted as unsigned integers.}
  \input{off8-restrictions.tex}
  \begin{operation}\off{8}\jump{\K{not} C}\end{operation}
  \begin{remarks}This instruction has the same encoding as \insnref{BC0}.\end{remarks}
\end{instruction}
 % alias
\begin{instruction}{BNC}{Branch if Not Carry}
  \begin{encoding*}{short}
    \mnemonic & \op{4}{1011} & \op{4}{0010} & \off{8} \\
  \end{encoding*}
  \begin{encoding*}{long}
    \exti
    \mnemonic & \op{4}{1011} & \op{4}{0010} & \off{8} \\
  \end{encoding*}
  \assembly{\mnemonic{} label}
  \purpose{To transfer control if the last arithmetic operation did not result in unsigned overflow.}
  \input{off8-restrictions.tex}
  \begin{operation}\off{8}\jump{\K{not} C}\end{operation}
  \begin{remarks}This instruction has the same encoding as \insnref{BC0}.\end{remarks}
\end{instruction}

\begin{instruction}{BNE}{Branch if Not Equal}
  \begin{encoding*}{short}
    \mnemonic & \op{4}{1011} & \op{4}{0000} & \off{8} \\
  \end{encoding*}
  \begin{encoding*}{long}
    \exti
    \mnemonic & \op{4}{1011} & \op{4}{0000} & \off{8} \\
  \end{encoding*}
  \assembly{\mnemonic{} label}
  \purpose{To transfer control after a \texttt{\insnref{CMP} Ra, Rb} instruction if \texttt{Ra} is not equal to \texttt{Rb}.}
  \input{off8-restrictions.tex}
  \begin{operation}\off{8}\jump{\K{not} Z}\end{operation}
  \begin{remarks}This instruction has the same encoding as \insnref{BZ0}.\end{remarks}
\end{instruction}
 % alias
\begin{instruction}{BNS}{Branch if Not Sign}
  \begin{encoding*}{short}
    \mnemonic & \op{4}{1011} & \op{4}{0001} & \off{8} \\
  \end{encoding*}
  \begin{encoding*}{long}
    \exti
    \mnemonic & \op{4}{1011} & \op{4}{0001} & \off{8} \\
  \end{encoding*}
  \assembly{\mnemonic{} label}
  \purpose{To transfer control if the last arithmetic operation did not produce a negative result.}
  \input{off8-restrictions.tex}
  \begin{operation}\off{8}\jump{\K{not} S}\end{operation}
  \begin{remarks}This instruction has the same encoding as \insnref{BS0}.\end{remarks}
\end{instruction}

\begin{instruction}{BNV}{Branch if Not Overflow}
  \begin{encoding*}{short}
    \mnemonic & \op{4}{1011} & \op{4}{0011} & \off{8} \\
  \end{encoding*}
  \begin{encoding*}{long}
    \exti
    \mnemonic & \op{4}{1011} & \op{4}{0011} & \off{8} \\
  \end{encoding*}
  \assembly{\mnemonic{} label}
  \purpose{To transfer control if the last arithmetic operation did not result in signed overflow.}
  \input{off8-restrictions.tex}
  \begin{operation}\off{8}\jump{\K{not} V}\end{operation}
  \begin{remarks}This instruction has the same encoding as \insnref{BV0}.\end{remarks}
\end{instruction}

\begin{instruction}{BNZ}{Branch if Not Zero}
  \begin{encoding*}{short}
    \mnemonic & \op{4}{1011} & \op{4}{0000} & \off{8} \\
  \end{encoding*}
  \begin{encoding*}{long}
    \exti
    \mnemonic & \op{4}{1011} & \op{4}{0000} & \off{8} \\
  \end{encoding*}
  \assembly{\mnemonic{} label}
  \purpose{To transfer control if the last operation produced a result not equal to zero.}
  \input{off8-restrictions.tex}
  \begin{operation}\off{8}\jump{\K{not} Z}\end{operation}
  \begin{remarks}This instruction has the same encoding as \insnref{BZ0}.\end{remarks}
\end{instruction}

\begin{instruction}{BS}{Branch if Negative}
  \begin{encoding*}{short}
    \mnemonic & \op{4}{1011} & \op{4}{1001} & \off{8} \\
  \end{encoding*}
  \begin{encoding*}{long}
    \exti
    \mnemonic & \op{4}{1011} & \op{4}{1001} & \off{8} \\
  \end{encoding*}
  \assembly{\mnemonic{} label}
  \purpose{To transfer control if the last arithmetic operation produced a negative result.}
  \input{off8-restrictions.tex}
  \begin{operation}\off{8}\jump{S}\end{operation}
  \begin{remarks}This instruction has the same encoding as \insnref{BS1}.\end{remarks}
\end{instruction}

\begin{instruction}{BS0}{Branch if Negative is 0}
  \begin{encoding*}{short}
    \mnemonic & \op{4}{1011} & \op{4}{0001} & \off{8} \\
  \end{encoding*}
  \begin{encoding*}{long}
    \exti
    \mnemonic & \op{4}{1011} & \op{4}{0001} & \off{8} \\
  \end{encoding*}
  \assembly{\mnemonic{} label}
  \purpose{To transfer control if the negative (S) flag is 0.}
  \input{off8-restrictions.tex}
  \begin{operation}\off{8}\jump{\K{not} S}\end{operation}
\end{instruction}

\begin{instruction}{BS1}{Branch if Negative is 1}
  \begin{encoding*}{short}
    \mnemonic & \op{4}{1011} & \op{4}{1001} & \off{8} \\
  \end{encoding*}
  \begin{encoding*}{long}
    \exti
    \mnemonic & \op{4}{1011} & \op{4}{1001} & \off{8} \\
  \end{encoding*}
  \assembly{\mnemonic{} label}
  \purpose{To transfer control if the negative (S) flag is 1.}
  \input{off8-restrictions.tex}
  \begin{operation}\off{8}\jump{S}\end{operation}
\end{instruction}

\begin{instruction}{BV}{Branch if Overflow}
  \begin{encoding*}{short}
    \mnemonic & \op{4}{1011} & \op{4}{1011} & \off{8} \\
  \end{encoding*}
  \begin{encoding*}{long}
    \exti
    \mnemonic & \op{4}{1011} & \op{4}{1011} & \off{8} \\
  \end{encoding*}
  \assembly{\mnemonic{} label}
  \purpose{To transfer control if the last arithmetic operation resulted in signed overflow.}
  \input{off8-restrictions.tex}
  \begin{operation}\off{8}\jump{V}\end{operation}
  \begin{remarks}This instruction has the same encoding as \insnref{BV1}.\end{remarks}
\end{instruction}

\begin{instruction}{BV0}{Branch if Overflow is 0}
  \begin{encoding*}{short}
    \mnemonic & \op{4}{1011} & \op{4}{0011} & \off{8} \\
  \end{encoding*}
  \begin{encoding*}{long}
    \exti
    \mnemonic & \op{4}{1011} & \op{4}{0011} & \off{8} \\
  \end{encoding*}
  \assembly{\mnemonic{} label}
  \purpose{To transfer control if the overflow (V) flag is 0.}
  \input{off8-restrictions.tex}
  \begin{operation}\off{8}\jump{\K{not} V}\end{operation}
\end{instruction}

\begin{instruction}{BV1}{Branch if Overflow is 1}
  \begin{encoding*}{short}
    \mnemonic & \op{4}{1011} & \op{4}{1011} & \off{8} \\
  \end{encoding*}
  \begin{encoding*}{long}
    \exti
    \mnemonic & \op{4}{1011} & \op{4}{1011} & \off{8} \\
  \end{encoding*}
  \assembly{\mnemonic{} label}
  \purpose{To transfer control if the overflow (V) flag is 1.}
  \input{off8-restrictions.tex}
  \begin{operation}\off{8}\jump{V}\end{operation}
\end{instruction}

\begin{instruction}{BZ}{Branch if Zero}
  \begin{encoding*}{short}
    \mnemonic & \op{4}{1011} & \op{4}{1000} & \off{8} \\
  \end{encoding*}
  \begin{encoding*}{long}
    \exti
    \mnemonic & \op{4}{1011} & \op{4}{1000} & \off{8} \\
  \end{encoding*}
  \assembly{\mnemonic{} label}
  \purpose{To transfer control if the last operation produced a result equal to zero.}
  \input{off8-restrictions.tex}
  \begin{operation}\off{8}\jump{Z}\end{operation}
  \begin{remarks}This instruction has the same encoding as \insnref{BZ1}.\end{remarks}
\end{instruction}

\begin{instruction}{BZ0}{Branch if Zero is 0}
  \begin{encoding*}{short}
    \mnemonic & \op{4}{1011} & \op{4}{0000} & \off{8} \\
  \end{encoding*}
  \begin{encoding*}{long}
    \exti
    \mnemonic & \op{4}{1011} & \op{4}{0000} & \off{8} \\
  \end{encoding*}
  \assembly{\mnemonic{} label}
  \purpose{To transfer control if the zero (Z) flag is 0.}
  \input{off8-restrictions.tex}
  \begin{operation}\off{8}\jump{\K{not} Z}\end{operation}
\end{instruction}

\begin{instruction}{BZ1}{Branch if Zero is 1}
  \begin{encoding*}{short}
    \mnemonic & \op{4}{1011} & \op{4}{1000} & \off{8} \\
  \end{encoding*}
  \begin{encoding*}{long}
    \exti
    \mnemonic & \op{4}{1011} & \op{4}{1000} & \off{8} \\
  \end{encoding*}
  \assembly{\mnemonic{} label}
  \purpose{To transfer control if the zero (Z) flag is 1.}
  \input{off8-restrictions.tex}
  \begin{operation}\off{8}\jump{Z}\end{operation}
\end{instruction}

\begin{instruction}{CMP}{Compare to Register}
  \begin{encoding}
    \mnemonic & \op{5}{00000} & \op{3}{000} & \reg{a} & \op{2}{11} & \reg{b} \\
  \end{encoding}
  \assembly{\mnemonic{} Ra, Rb}
  \purpose{To compare 16-bit two's complement integers in registers.}
  \restrictions{None.}
  \begin{operation}\aluRR{opA + \K{not} opB + 1}\flagZSBV\end{operation}
  \begin{remarks}This instruction behaves identically to \texttt{SUB}, with the exception that it discards the computed value.\end{remarks}
\end{instruction}

\begin{instruction}{CMPI}{Compare to Immediate}
  \begin{encoding*}{short}
    \mnemonic & \op{5}{00001} & \opx{3}{000} & \reg{a} & \op{2}{11} & \imm{3} \\
  \end{encoding*}
  \begin{encoding*}{long}
    \exti
    \mnemonic & \op{5}{00001} & \opx{3}{000} & \reg{a} & \op{2}{11} & \imm{3} \\
  \end{encoding*}
  \assembly{\mnemonic{} Rd, Ra, imm}
  \purpose{To compare a two's complement constant to a 16-bit two's complement integer in a register.}
  \restrictions{None.}
  \begin{operation}\aluRI{al}{opA + \K{not} opB + 1}\flagZSBV\end{operation}
  \begin{remarks}This instruction behaves identically to \texttt{SUBI}, with the exception that it discards the computed value.\end{remarks}
\end{instruction}

\input{EXTI.tex}
\begin{instruction}{J}{Jump}
  \begin{encoding*}{short}
    \mnemonic & \op{4}{1011} & \op{4}{1111} & \off{8} \\
  \end{encoding*}
  \begin{encoding*}{long}
    \exti
    \mnemonic & \op{4}{1011} & \op{4}{1111} & \off{8} \\
  \end{encoding*}
  \assembly{\mnemonic{} label}
  \purpose{To unconditionally transfer control.}
  \input{off8-restrictions.tex}
  \begin{operation}\off{8}PC ← PC + 1 + off\end{operation}
\end{instruction}

\begin{instruction}{JAL}{Jump and Link}
  \begin{encoding*}{short}
    \mnemonic & \op{5}{10101} & \reg{d} & \off{8} \\
  \end{encoding*}
  \begin{encoding*}{long}
    \exti
    \mnemonic & \op{5}{10101} & \reg{d} & \off{8} \\
  \end{encoding*}
  \assembly{\mnemonic{} Rd, label}
  \purpose{To transfer control to a subroutine.}
  \input{off8-restrictions.tex}
\begin{operation}\off{8}
mem[W+Rd] ← PC + 1
PC ← PC + 1 + off
\end{operation}
\end{instruction}

\begin{instruction}{JR}{Jump to Register}
  \begin{encoding*}{short}
    \mnemonic & \op{5}{10100} & \reg{s} & \op{3}{100} & \off{5} \\
  \end{encoding*}
  \begin{encoding*}{long}
    \exti
    \mnemonic & \op{5}{10100} & \reg{s} & \op{3}{100} & \off{5} \\
  \end{encoding*}
  \assembly{\mnemonic{} Rs, off}
  \purpose{To transfer control to a variable absolute address contained in a register, with a constant offset.}
  \input{off5-restrictions.tex}
\begin{operation}\off{5}
PC ← (mem[W+Ra] + off)[15:0]
\end{operation}
\end{instruction}

\begin{instruction}{JRAL}{Jump to Register and Link}
  \begin{encoding}
    \mnemonic & \op{5}{10100} & \reg{d} & \op{3}{101} & \opx{2}{00} & \reg{b} \\
  \end{encoding}
  \assembly{\mnemonic{} Rd, Rb}
  \purpose{To transfer control to a subroutine whose variable absolute address is contained in a register.}
  \restrictions{None.}
\begin{operation}
addr ← mem[W+Rb]
mem[W+Rd] ← (PC + 1)[15:0]
PC ← addr
\end{operation}
\end{instruction}

\begin{instruction}{JST}{Jump through Switch Table}
  \begin{encoding*}{short}
    \mnemonic & \op{5}{10100} & \reg{s} & \op{3}{111} & \off{5} \\
  \end{encoding*}
  \begin{encoding*}{long}
    \exti
    \mnemonic & \op{5}{10100} & \reg{s} & \op{3}{111} & \off{5} \\
  \end{encoding*}
  \assembly{\mnemonic{} Rs, off}
  \purpose{To transfer control to an address contained in a jump table at a variable offset, where the address is relative to the location of the table.}
  \input{off5-restrictions.tex}
\begin{operation}\off{5}
table ← PC + 1 + off
entry ← mem[W+Rs]
addr ← mem[table + entry]
PC ← (table + addr)[15:0]
\end{operation}
\end{instruction}

\begin{instruction}{JVT}{Jump through Virtual Table}
  \begin{encoding*}{short}
    \mnemonic & \op{5}{10100} & \reg{s} & \op{3}{110} & \off{5} \\
  \end{encoding*}
  \begin{encoding*}{long}
    \exti
    \mnemonic & \op{5}{10100} & \reg{s} & \op{3}{110} & \off{5} \\
  \end{encoding*}
  \assembly{\mnemonic{} Rs, off}
  \purpose{To transfer control to an address contained in a jump table at a constant offset, where the address is relative to the location of the table.}
  \input{off5-restrictions.tex}
\begin{operation}\off{5}
table ← mem[W+Rs]
addr ← mem[table + off]
PC ← (table + addr)[15:0]
\end{operation}
\end{instruction}

\begin{instruction}{LD}{Load}
  \begin{encoding*}{short}
    \mnemonic & \op{5}{01000} & \reg{d} & \reg{a} & \off{5} \\
  \end{encoding*}
  \begin{encoding*}{long}
    \exti
    \mnemonic & \op{5}{01000} & \reg{d} & \reg{a} & \off{5} \\
  \end{encoding*}
  \assembly{\mnemonic{} Rd, Ra, off}
  \purpose{To load a word from memory at a variable address, with a constant offset.}
  \input{off5-restrictions.tex}
\begin{operation}\off{5}
addr ← mem[W+Ra] + off
data ← mem[addr]
mem[W+Rd] ← data
\end{operation}
\end{instruction}

\begin{instruction}{LDR}{Load PC-relative}
  \begin{encoding*}{short}
    \mnemonic & \op{5}{01001} & \reg{d} & \reg{a} & \off{5} \\
  \end{encoding*}
  \begin{encoding*}{long}
    \exti
    \mnemonic & \op{5}{01001} & \reg{d} & \reg{a} & \off{5} \\
  \end{encoding*}
  \assembly{\mnemonic{} Rd, Ra, off}
  \purpose{To load a word from memory at a constant PC-relative address, with a variable offset.}
  \input{off5-restrictions.tex}
\begin{operation}\off{5}
addr ← PC + 1 + off + mem[W+Ra]
data ← mem[addr]
mem[W+Rd] ← data
\end{operation}
\end{instruction}

\begin{instruction}{LDW}{Adjust and Load Window Address}
  \begin{encoding*}{short}
    \mnemonic & \op{5}{10100} & \reg{d} & \op{3}{011} & \imm{5} \\
  \end{encoding*}
  \begin{encoding*}{long}
    \exti
    \mnemonic & \op{5}{10100} & \reg{d} & \op{3}{011} & \imm{5} \\
  \end{encoding*}
  \assembly{\mnemonic{} Rd, imm}
  \purpose{To increase or decrease the address of the register window, and retrieve the prior address of the register window.}
  \restrictions{If \texttt{imm} contains a value that is not a multiple of 8, the behavior is \unpredictable. If the long form is used, and \texttt{imm5[4:3]} are non-zero, the behavior is \unpredictable.}
\begin{operation}\imm{5}
old ← W
W ← W + imm
mem[W+Rd] ← old
\end{operation}
  \begin{remarks}See also \insnref{STW}. This instruction may be used in a function prologue, where \texttt{Rd} is any register chosen to act as a frame pointer.\end{remarks}
  \begin{notice}The interpretation of the immediate field of this instruction is not final.\end{notice}
\end{instruction}

\begin{instruction}{LDX}{Load External}
  \begin{encoding*}{short}
    \mnemonic & \op{5}{01100} & \reg{d} & \reg{a} & \off{5} \\
  \end{encoding*}
  \begin{encoding*}{long}
    \exti
    \mnemonic & \op{5}{01100} & \reg{d} & \reg{a} & \off{5} \\
  \end{encoding*}
  \assembly{\mnemonic{} Rd, Ra, off}
  \purpose{To complete a load cycle on external bus at a variable address, with a constant offset.}
  \input{off5-restrictions.tex}
\begin{operation}\off{5}
addr ← mem[W+Ra] + off
data ← ext[addr]
mem[W+Rd] ← data
\end{operation}
\end{instruction}

\begin{instruction}{LDXA}{Load External Absolute}
  \begin{encoding*}{short}
    \mnemonic & \op{5}{01101} & \reg{d} & \off{8} \\
  \end{encoding*}
  \begin{encoding*}{long}
    \exti
    \mnemonic & \op{5}{01101} & \reg{d} & \off{8} \\
  \end{encoding*}
  \assembly{\mnemonic{} Rd, off}
  \purpose{To complete a load cycle on external bus at a constant absolute address.}
  \input{off8-restrictions.tex}
\begin{operation}\off{8}
data ← ext[off]
mem[W+Rd] ← data
\end{operation}
\end{instruction}

\input{MOV.tex} % pseudo
\begin{instruction}{MOVI}{Move Immediate}
  \begin{encoding*}{short}
    \mnemonic & \op{5}{10000} & \reg{d} & \imm{8} \\
  \end{encoding*}
  \begin{encoding*}{long}
    \exti
    \mnemonic & \op{5}{10000} & \reg{d} & \imm{8} \\
  \end{encoding*}
  \assembly{\mnemonic{} Rd, imm}
  \purpose{To load a register with a constant.}
  \restrictions{If the long form is used, and \texttt{imm8[8:3]} are non-zero, the behavior is \unpredictable.}
\begin{operation}\imm{8}
mem[W+Rd] ← imm
\end{operation}
\end{instruction}

\begin{instruction}{MOVR}{Move PC-relative Address}
  \begin{encoding*}{short}
    \mnemonic & \op{5}{10001} & \reg{d} & \off{8} \\
  \end{encoding*}
  \begin{encoding*}{long}
    \exti
    \mnemonic & \op{5}{10001} & \reg{d} & \off{8} \\
  \end{encoding*}
  \assembly{\mnemonic{} Rd, off}
  \purpose{To load a register with an address relative to PC with a constant offset.}
  \input{off8-restrictions.tex}
\begin{operation}\off{8}
mem[W+Rd] ← (PC + 1 + off)[15:0]
\end{operation}
\end{instruction}

\begin{instruction}{NOP}{No Operation}
  \begin{encoding}
    \mnemonic & \op{4}{1011} & \op{4}{0111} & \op{8}{00000000} \\
  \end{encoding}
  \assembly{\mnemonic{}}
  \purpose{To perform no operation while consuming one instruction word.}
  \begin{operation}PC ← PC + 1\end{operation}
  \begin{remarks}The \texttt{NOP} instruction is encoded as a conditional branch, whose condition is always false, that targets the next instruction.\end{remarks}
\end{instruction}

\begin{instruction}{NOT}{Logical NOT}
  \assembly{\mnemonic{} Rd, Rs}
  \purpose{To invert all the bits in a register.}
  \restrictions{None.}
\begin{remarks}
The translates to a XORI with an 0xFFFF immediate 
\begin{alltt}
    XORI Rd, Rs, 0xFFFF
\end{alltt}
\end{remarks}
\end{instruction}
 % pseudo
\input{OR.tex}
\input{ORI.tex}
\begin{instruction}{ROL}{Rotate Left}
  \begin{encoding}
    \mnemonic & \op{5}{00100} & \reg{d} & \reg{a} & \op{2}{01} & \reg{b} \\
  \end{encoding}
  \assembly{\mnemonic{} Rd, Ra, Rb}
  \purpose{To perform a left rotate of a 16-bit integer in a register by a variable bit amount.}
  \input{shift-restrictions.tex}
  \begin{operation}\aluRR{\string{opA[15-opB:0], opA[15:16-opB]\string}}\wb\flagZS\end{operation}
\end{instruction}

\begin{instruction}{ROLI}{Rotate Left Immediate}
  \begin{encoding}
    \mnemonic & \op{5}{00101} & \reg{d} & \reg{a} & \op{2}{01} & \imm{3} \\
  \end{encoding}
  \assembly{\mnemonic{} Rd, Ra, amount}
  \purpose{To perform a left rotate of a 16-bit integer in a register by a constant bit amount.}
  \restrictions{If \texttt{amount} is greater than 15, the behavior is \unpredictable.}

  \begin{operation}\aluRI{sr}{\string{opA[15-opB:0], opA[15:15-opB]\string}}\wb\flagZS\end{operation}
\end{instruction}

\input{RORI.tex} % pseudo
\begin{instruction}{SBC}{Subtract Register with Carry}
  \begin{encoding}
    \mnemonic & \op{5}{00010} & \reg{d} & \reg{a} & \op{2}{11} & \reg{b} \\
  \end{encoding}
  \assembly{\mnemonic{} Rd, Ra, Rb}
  \purpose{To subtract 16-bit two's complement integers in registers, with carry input. If the carry input is 1, the previous operation did not borrow.}
  \restrictions{None.}
  \begin{operation}\aluRR{opA + \K{not} opB + C}\wb\flagZSBV\end{operation}
\begin{remarks}
A 32-bit subtraction with both operands in registers can be performed as follows:
\begin{alltt}
; Perform \string{R1, R0\string} ← \string{R3, R2\string} - \string{R5, R4\string}
    SUB  R0, R2, R4
    SBC  R1, R3, R5
\end{alltt}
\end{remarks}
\end{instruction}

\begin{instruction}{SBCI}{Subtract Immediate with Carry}
  \begin{encoding*}{short}
    \mnemonic & \op{5}{00011} & \reg{d} & \reg{a} & \op{2}{11} & \imm{3} \\
  \end{encoding*}
  \begin{encoding*}{long}
    \exti
    \mnemonic & \op{5}{00011} & \reg{d} & \reg{a} & \op{2}{11} & \imm{3} \\
  \end{encoding*}
  \assembly{\mnemonic{} Rd, Ra, imm}
  \purpose{To subtract a two's complement constant from a 16-bit two's complement integer in a register, with carry input. If the carry input is 1, the previous operation did not borrow.}
  \restrictions{None.}
  \begin{operation}\aluRI{al}{opA + \K{not} opB + C}\wb\flagZSBV\end{operation}
\begin{remarks}
A 32-bit subtraction with a register and an immediate operand can be performed as follows:
\begin{alltt}
; Perform \string{R1, R0\string} ← \string{R3, R2\string} - 0x40001
    SUBI R0, R2, 1
    SBCI R1, R3, 4
\end{alltt}
\end{remarks}
\end{instruction}

\begin{instruction}{SLL}{Shift Left Logical}
  \begin{encoding}
    \mnemonic & \op{5}{00100} & \reg{d} & \reg{a} & \op{2}{00} & \reg{b} \\
  \end{encoding}
  \assembly{\mnemonic{} Rd, Ra, Rb}
  \purpose{To perform a left logical shift of a 16-bit integer in a register by a variable bit amount.}
  \input{shift-restrictions.tex}
  \begin{operation}\aluRR{\string{opA[15-opB:0], \string{opB\string{0\string}\string}\string}}\wb\flagZS\end{operation}
\end{instruction}

\begin{instruction}{SLLI}{Shift Left Logical Immediate}
  \begin{encoding*}{short}
    \mnemonic & \op{5}{00101} & \reg{d} & \reg{a} & \op{2}{00} & \imm{3} \\
  \end{encoding*}
  \begin{encoding*}{long}
    \exti
    \mnemonic & \op{5}{00101} & \reg{d} & \reg{a} & \op{2}{00} & \imm{3} \\
  \end{encoding*}
  \assembly{\mnemonic{} Rd, Ra, amount}
  \purpose{To perform a left logical shift of a 16-bit integer in a register by a constant bit amount.}
  \restrictions{If \texttt{amount} is greater than 15, the behavior is \unpredictable.}

  \begin{operation}\aluRI{sr}{\string{opA[15-opB:0], opB\string{0\string}\string}}\wb\flagZS\end{operation}
\end{instruction}

\begin{instruction}{SRA}{Shift Right Arithmetical}
  \begin{encoding}
    \mnemonic & \op{5}{00100} & \reg{d} & \reg{a} & \op{2}{11} & \reg{b} \\
  \end{encoding}
  \assembly{\mnemonic{} Rd, Ra, Rb}
  \purpose{To perform a right arithmetical shift of a 16-bit integer in a register by a variable bit amount.}
  \input{shift-restrictions.tex}
  \begin{operation}\aluRR{\string{\string{opB\string{opA[15]\string}\string}, opA[15:opB]\string}}\wb\flagZS\end{operation}
\end{instruction}

\begin{instruction}{SRAI}{Shift Right Arithmetical Immediate}
  \begin{encoding}
    \mnemonic & \op{5}{00100} & \reg{d} & \reg{a} & \op{2}{11} & \imm{3} \\
  \end{encoding}
  \assembly{\mnemonic{} Rd, Ra, amount}
  \purpose{To perform a right arithmetical shift of a 16-bit integer in a register by a constant bit amount.}
  \restrictions{If \texttt{amount} is greater than 15, the behavior is \unpredictable.}

  \begin{operation}\aluRI{sr}{\string{\string{opB\string{opA[15]\string}\string}, opA[15:opB]\string}}\wb\flagZS\end{operation}
\end{instruction}

\begin{instruction}{SRL}{Shift Right Logical}
  \begin{encoding}
    \mnemonic & \op{5}{00100} & \reg{d} & \reg{a} & \op{2}{10} & \reg{b} \\
  \end{encoding}
  \assembly{\mnemonic{} Rd, Ra, Rb}
  \purpose{To perform a right logical shift of a 16-bit integer in a register by a variable bit amount.}
  \input{shift-restrictions.tex}
  \begin{operation}\aluRR{\string{\string{opB\string{0\string}\string}, opA[15:opB]\string}}\wb\flagZS\end{operation}
\end{instruction}

\begin{instruction}{SRLI}{Shift Right Logical Immediate}
  \begin{encoding}
    \mnemonic & \op{5}{00101} & \reg{d} & \reg{a} & \op{2}{10} & \imm{3} \\
  \end{encoding}
  \assembly{\mnemonic{} Rd, Ra, amount}
  \purpose{To perform a right logical shift of a 16-bit integer in a register by a constant bit amount.}
  \restrictions{If \texttt{amount} is greater than 15, the behavior is \unpredictable.}

  \begin{operation}\aluRI{sr}{\string{\string{opB\string{opA[15]\string}\string}, opA[15:opB]\string}}\wb\flagZS\end{operation}
\end{instruction}

\begin{instruction}{ST}{Store}
  \begin{encoding*}{short}
    \mnemonic & \op{5}{01010} & \reg{s} & \reg{a} & \off{5} \\
  \end{encoding*}
  \begin{encoding*}{long}
    \exti
    \mnemonic & \op{5}{01010} & \reg{s} & \reg{a} & \off{5} \\
  \end{encoding*}
  \assembly{\mnemonic{} Rs, Ra, off}
  \purpose{To store a word to memory at a variable address, with a constant offset.}
  \input{off5-restrictions.tex}
\begin{operation}\off{5}
addr ← mem[W+Ra] + off
data ← mem[W+Rs]
mem[addr] ← data
\end{operation}
\end{instruction}

\begin{instruction}{STR}{Store PC-relative}
  \begin{encoding*}{short}
    \mnemonic & \op{5}{01011} & \reg{s} & \reg{a} & \off{5} \\
  \end{encoding*}
  \begin{encoding*}{long}
    \exti
    \mnemonic & \op{5}{01011} & \reg{s} & \reg{a} & \off{5} \\
  \end{encoding*}
  \assembly{\mnemonic{} Rs, Ra, off}
  \purpose{To store a word to memory at a constant PC-relative address, with a variable offset.}
  \input{off5-restrictions.tex}
\begin{operation}\off{5}
addr ← PC + 1 + off + mem[W+Ra]
data ← mem[W+Rs]
mem[addr] ← data
\end{operation}
\end{instruction}

\begin{instruction}{STW}{Store to Window Address}
  \begin{encoding}
    \mnemonic & \op{5}{10100} & \opx{3}{000} & \op{3}{000} & \opx{2}{00} & \reg{b} \\
  \end{encoding}
  \assembly{\mnemonic{} Rb}
  \purpose{To arbitrarily change the address of the register window.}
  \restrictions{If \texttt{Rb} contains a value that is not a multiple of 8, the behavior is \unpredictable.}
\begin{operation}
W ← mem[W+Rb]
\end{operation}
  \begin{remarks}See also \insnref{LDW}. This instruction may be used in a function epilogue, where \texttt{Rb} is any register chosen to act as a frame pointer.\end{remarks}
\end{instruction}

\begin{instruction}{STX}{Store External}
  \begin{encoding*}{short}
    \mnemonic & \op{5}{01110} & \reg{s} & \reg{a} & \off{5} \\
  \end{encoding*}
  \begin{encoding*}{long}
    \exti
    \mnemonic & \op{5}{01110} & \reg{s} & \reg{a} & \off{5} \\
  \end{encoding*}
  \assembly{\mnemonic{} Rs, Ra, off}
  \purpose{To complete a store cycle on external bus at a variable address, with a constant offset.}
  \input{off5-restrictions.tex}
\begin{operation}\off{5}
addr ← mem[W+Ra] + off
data ← mem[W+Rs]
ext[addr] ← data
\end{operation}
\end{instruction}

\begin{instruction}{STXA}{Store External Absolute}
  \begin{encoding*}{short}
    \mnemonic & \op{5}{01111} & \reg{s} & \off{8} \\
  \end{encoding*}
  \begin{encoding*}{long}
    \exti
    \mnemonic & \op{5}{01111} & \reg{s} & \off{8} \\
  \end{encoding*}
  \assembly{\mnemonic{} Rs, off}
  \purpose{To complete a store cycle on external bus at a constant absolute address.}
  \input{off8-restrictions.tex}
\begin{operation}\off{8}
data ← mem[W+Rs]
ext[off] ← data
\end{operation}
\end{instruction}

\begin{instruction}{SUB}{Subtract Register}
  \begin{encoding}
    \mnemonic & \op{5}{00010} & \reg{d} & \reg{a} & \op{2}{10} & \reg{b} \\
  \end{encoding}
  \assembly{\mnemonic{} Rd, Ra, Rb}
  \purpose{To subtract 16-bit two's complement integers in registers.}
  \restrictions{None.}
  \begin{operation}\aluRR{opA + \K{not} opB + 1}\wb\flagZSBV\end{operation}
\end{instruction}

\begin{instruction}{SUBI}{Subtract Immediate}
  \begin{encoding*}{short}
    \mnemonic & \op{5}{00011} & \reg{d} & \reg{a} & \op{2}{10} & \imm{3} \\
  \end{encoding*}
  \begin{encoding*}{long}
    \exti
    \mnemonic & \op{5}{00011} & \reg{d} & \reg{a} & \op{2}{10} & \imm{3} \\
  \end{encoding*}
  \assembly{\mnemonic{} Rd, Ra, imm}
  \purpose{To subtract a two's complement constant from a 16-bit two's complement integer in a register.}
  \restrictions{None.}
  \begin{operation}\aluRI{al}{opA + \K{not} opB + 1}\wb\flagZSBV\end{operation}
\end{instruction}

\input{XCHG.tex} % pseudo
\begin{instruction}{XCHW}{Exchange Window Address}
  \begin{encoding}
    \mnemonic & \op{5}{10100} & \reg{d} & \op{3}{001} & \opx{2}{00} & \reg{b} \\
  \end{encoding}
  \assembly{\mnemonic{} Rd, Rb}
  \purpose{To exchange the address of the register window with a general purpose register.}
  \restrictions{If \texttt{Rb} contains a value that is not a multiple of 8, the behavior is \unpredictable.}
\begin{operation}
old ← W
W ← mem[W+Rb]
mem[W+Rd] ← old
\end{operation}
\begin{remarks}
This instruction may be used in a context switch routine. For example, if multiple register windows are set up such that each contains the address of the next one in \texttt{R7}, the following code may be used to switch contexts:
\begin{alltt}
yield:
    XCHW R7, R7
    JR   R0
; Elsewhere:
    JAL R0, yield
\end{alltt}
\end{remarks}
\end{instruction}

\input{XOR.tex}
\input{XORI.tex}
